\documentclass[11pt]{article}
\setlength{\topmargin}{-0.5in}
\setlength{\textheight}{9in}
\setlength{\oddsidemargin}{0.25in}
\setlength{\textwidth}{6in}
\usepackage{mathpazo}
\usepackage[T1]{fontenc}
\usepackage[utf8]{inputenc}

\usepackage{amsmath}
\usepackage{amsthm}
\usepackage{amssymb}
\usepackage{appendix}
\usepackage{array}
\usepackage{bm}
\usepackage{cancel}
\usepackage{cite}
\usepackage{courier}
\usepackage{graphicx}
\usepackage{empheq}
\usepackage{enumitem}
\usepackage{listings}
\usepackage{mathtools}
\usepackage{units}
\usepackage{bigstrut}
\usepackage{rotating}
\usepackage{ mathrsfs }
\usepackage{multirow}
\usepackage{booktabs}
\usepackage{setspace}

\pagenumbering{gobble}
\renewcommand{\baselinestretch}{1.5}

\usepackage{floatrow}
\floatsetup[figure]{capposition=top}
\DeclareMathAlphabet{\mathcal}{OMS}{cmsy}{m}{n}

\DeclareMathAlphabet{\mathsfit}{\encodingdefault}{\sfdefault}{m}{}
\SetMathAlphabet{\mathsfit}{bold}{\encodingdefault}{\sfdefault}{bx}{}

\newcommand{\tens}[1]{\bm{\mathsfit{#1}}}

\usepackage{color}
\lstset{language=R,basicstyle=\ttfamily,breaklines=true,
                keywordstyle=\color{blue}\ttfamily,
                stringstyle=\color{red}\ttfamily,
                commentstyle=\color{magenta}\ttfamily,
                showstringspaces=false,
                }

\newcommand*\widefbox[1]{\fbox{\hspace{2em}#1\hspace{2em}}}
\newcommand*\mb{\mathbf}
\newcommand*\reals{\mathbb{R}}
\newcommand*\complex{\mathbb{C}}
\newcommand*\naturals{\mathbb{N}}
\newcommand*\nats{\naturals}
\newcommand*\integers{\mathbb{Z}}
\newcommand*\rationals{\mathbb{Q}}
\newcommand*\irrationals{\mathbb{J}}
\newcommand*\pd{\partial}
\newcommand*\htab{\hspace{4 mm}}
\newcommand*\vtab{\vspace{0.5 in}}
\newcommand*\lsent{\mathcal{L}}
\newcommand*\conj{\overline}
\newcommand*\union{\cup}
\newcommand*\intersect{\cap}
\newcommand*\cl{\cancel}
\newcommand*\ANS{\text{ANS}}
\newcommand*\As{\text{As}}
\newcommand*\then{\rightarrow}
\newcommand*\elim{\text{E}}
\newcommand*\intro{\text{I}}
\newcommand*\absurd{\curlywedge}
\newcommand*\NK{\vdash_{\text{NK}}}
\newcommand*\derivation{\begin{tabular} { >{$}l<{$}  >{$}c<{$}  >{$}l<{$}  >{$}r<{$} }}
\newcommand*\interp{\mathcal{I}}
\newcommand*\ba{\[ \begin{aligned}}
\newcommand*\ea{\end{aligned} \]}
\newcommand*\C{\mathcal{C}}
\newcommand*\card[1]{\text{card}\left(#1\right)}
\newcommand*\D{\mathscr{D}}
\newcommand*\df{=_{\text{def}}}
\newcommand*\eps{\epsilon}
\newcommand*\enum{\begin{enumerate}[label=(\alph*)]}
\newcommand*\enumend{\end{enumerate}}
\newcommand*\E[1]{\mathsf{E}\left[#1\right]}
\newcommand*\Esub[2]{\mathsf{E}_{#1}\left[#2\right]}
\newcommand*\Var[1]{\text{Var}\left[#1\right]}
\newcommand*\Cov[1]{\;\text{Cov}\left[#1\right]}
\newcommand*\iid{\overset{\text{iid}}{\sim}}
\newcommand*\Exp[1][\lambda]{\text{Exp}(\text{rate}=#1)}
\newcommand*\ind[1]{\mathbf{1}\left(#1\right)}
\newcommand*\set[1]{\left\{#1\right\}}
\newcommand*\estim[1]{\widehat{#1}}
\newcommand*\der{\text{d}}
\newcommand*\deriv[2]{\frac{\der #1}{\der #2}}
\newcommand*\pderiv[2]{\frac{\pd #1}{\pd #2}}
\newcommand*\norm[1]{\left\|#1\right\|}
\newcommand*\dist[2]{\;\text{dist}\left(#1, #2\right)}
\newcommand*\interior{\text{int}\;}
\newcommand*\exterior{\text{ext}\;}
\newcommand*\boundary{\text{bd}\;}
\newcommand*\lh{\overset{\text{L'H}}{=}}
\newcommand*\bA{\mathbf{A}}
\newcommand*\bb{\mathbf{b}}
\newcommand*\bB{\mathbf{B}}
\newcommand*\bc{\mathbf{c}}
\newcommand*\be{\mathbf{e}}
\newcommand*\bh{\mathbf{h}}
\newcommand*\bI{\mathbf{I}}
\newcommand*\bK{\mathbf{K}}
\newcommand*\bL{\mathbf{L}}
\newcommand*\bo{\mathbf{o}}
\newcommand*\br{\mathbf{r}}
\newcommand*\bs{\mathbf{s}}
\newcommand*\bS{\mathbf{S}}
\newcommand*\bt{\mathbf{t}}
\newcommand*\bu{\mathbf{u}}
\newcommand*\bv{\mathbf{v}}
\newcommand*\bx{\mathbf{x}}
\newcommand*\bw{\mathbf{w}}
\newcommand*\bW{\mathbf{W}}
\newcommand*\bX{\mathbf{X}}
\newcommand*\by{\mathbf{y}}
\newcommand*\bY{\mathbf{Y}}
\newcommand*\bZ{\mathbf{Z}}
\newcommand*\bz{\mathbf{z}}
\newcommand*\bp{\mathbf{p}}
\newcommand*\bzero{\mathbf{0}}
\newcommand*\bone{\mathbf{1}}
\newcommand*\balpha{\boldsymbol{\alpha}}
\newcommand*\bbeta{\boldsymbol{\beta}}
\newcommand*\bgamma{\boldsymbol{\gamma}}
\newcommand*\beps{\boldsymbol{\varepsilon}}
\newcommand*\btheta{\boldsymbol{\theta}}
\newcommand*\bTheta{\boldsymbol{\Theta}}
\newcommand*\bmu{\boldsymbol{\mu}}
\newcommand*\bsigma{\boldsymbol{\sigma}}
\newcommand*\bSigma{\boldsymbol{\Sigma}}
\newcommand*\bOmega{\boldsymbol{\Omega}}
\newcommand\Psub[2]{\tens{P}_{#1}\left[#2\right]}
\newcommand\e{\operatorname{e}}
\newcommand\prox{\operatorname{prox}}
\newcommand\T{\mathsf{T}}

\newread\tmp
\newcommand\getcount{
	\openin\tmp=knot_count.txt
	\read\tmp to \knots
	\closein\tmp
	
	\openin\tmp=span.txt
	\read\tmp to \spanval
	\closein\tmp
	
	\openin\tmp=span_g.txt
	\read\tmp to \spantwo
	\closein\tmp
	
	\openin\tmp=span_g_2.txt
	\read\tmp to \spanthree
	\closein\tmp
}


\renewcommand\Re{\operatorname{Re}}
\renewcommand\Im{\operatorname{Im}}
\DeclareMathOperator*{\argmin}{arg\;min}
\DeclareMathOperator*{\argmax}{arg\;max}
\renewcommand\;{\,}
\renewcommand\epsilon{\varepsilon}
\renewcommand\rho{\varrho}
\renewcommand\phi{\varphi}
\renewcommand\mod{\hspace{0.2em} \textbf{mod}\hspace{0.2em}}
\renewcommand\Pr[1]{ \mathsf{Pr}\left[#1\right] }

\lstset{breaklines=true,
        numbersep=5pt,
        xleftmargin=.25in,
        xrightmargin=.25in}

\DeclareMathOperator{\sech}{sech}
\DeclareMathOperator{\sgn}{sgn}
\makeatletter
\renewcommand*\env@matrix[1][*\c@MaxMatrixCols c]{%
  \hskip -\arraycolsep
  \let\@ifnextchar\new@ifnextchar
  \array{#1}}
\makeatother

\newenvironment{amatrix}[1]{%
  \left(\begin{array}{@{}*{#1}{c}|c@{}}
}{%
  \end{array}\right)
}
\vspace{-1in}

\begin{document}
\title{STAT 570: Homework 2}
\author{Branden Olson}
\date{}
\maketitle

\section*{Problem 1}
\begin{enumerate}[label=(\roman*)]
\item
$\eps_i \sim N(0, 3^2)$:
\\
We know that the OLS estimator of $\bbeta$ is
\ba
\estim{\bbeta} = (\bX^T \bX)^{-1} \bX^T \bY
\ea
and thus
\ba
\E{\estim{\bbeta}} & = (\bX^T \bX)^{-1} \bX^T \E{ \bX \bbeta + \beps} \\
	& = (\bX^T \bX)^{-1} \bX^T \bX \bbeta + (\bX^T \bX)^{-1} \bX^T \E{\beps} \\
	& = \bbeta + (\bX^T \bX)^{-1} \bX^T \cdot \bzero \\
	& = \bbeta
\ea
so that, as long as the error terms in our linear model are mean-zero, the OLS estimator will be unbiased. 
\item
Here, we have
\ba
\Var{\estim{\bbeta}} & = 
	(\bX^T \bX)^{-1} \bX^T \Var{\bY} \bX (\bX^T \bX)^{-T} \\
	& = (\bX^T \bX)^{-1} \bX^T \Var{\eps_1} \bI \bX (\bX^T \bX)^{-T} \\
	& = \Var{\eps_1} (\bX^T \bX)^{-1}
\ea
so that the variance of the OLS estimator will depend on the variance of the true errors. 
\begin{enumerate}[label=(\roman*)]
\item If $\eps_i \sim \mathcal N(0, 3^2)$, then
$\Var{\estim{\bbeta}} = 9 (\bX^T \bX)^{-1}$.
\item If $\eps_i \sim \mathcal U(-3, 3)$, then 
$\Var{\estim{\bbeta}} = \frac{(3 - (-3))^2}{12} (\bX^T \bX)^{-1}
= 3 (\bX^T \bX)^{-1}$
\item If 
$\eps_i \sim \text{SN}\left(\alpha=5, \omega=1, \xi=-\omega \delta \sqrt{2\over\pi}\right)$, with $\delta = {\alpha \over \sqrt{1 + \alpha^2}}$, then we see that
\ba
\E{\eps_i} = \xi + \omega \delta \sqrt{2\over\pi} = 0
\ea
\ba
\Var{\eps_i} & = \omega^2 \left(1 - \frac{2\delta^2}{\pi}\right) \\
	& = 1 - \frac{2 \left(\frac{5}{\sqrt{1 + 5^2}}\right)^2}{ \pi} \\
	& = 1 - \frac{2 \cdot 25}{\pi \cdot 26} \\
	& = 1 - \frac{25}{13\pi}  \htab (\approx 0.387)
\ea
So that $\Var{\bbeta} = \left(1 - \frac{25}{13\pi}\right) (\bX^T \bX)^{-1}$. 
\item
Below is a table of the theoretical and empirical variances for $\estim{\bbeta}$ for each distribution.
% latex table generated in R 3.4.1 by xtable 1.8-2 package
% Thu Oct 12 17:21:23 2017
\begin{table}[ht]
\centering
\begin{tabular}{rrrllr}
  \hline
 & $\Var{\bbeta_0}$ & $\Var{\bbeta_1}$ & Type & Distribution & $n$ \\ 
  \hline
1 & 50.86 & 0.122 & Theoretical & Normal & 10 \\ 
  2 & 51.54 & 0.124 & Empirical & Normal & 10 \\ 
  3 & 16.95 & 0.041 & Theoretical & Uniform & 10 \\ 
  4 & 17.02 & 0.041 & Empirical & Uniform & 10 \\ 
  5 & 2.19 & 0.005 & Theoretical & Skew-normal & 10 \\ 
  6 & 2.20 & 0.005 & Empirical & Skew-normal & 10 \\ 
  7 & 11.07 & 0.027 & Theoretical & Normal & 25 \\ 
  8 & 11.29 & 0.027 & Empirical & Normal & 25 \\ 
  9 & 3.69 & 0.009 & Theoretical & Uniform & 25 \\ 
  10 & 3.65 & 0.009 & Empirical & Uniform & 25 \\ 
  11 & 0.48 & 0.001 & Theoretical & Skew-normal & 25 \\ 
  12 & 0.47 & 0.001 & Empirical & Skew-normal & 25 \\ 
   \hline
\end{tabular}
\end{table}

\end{enumerate}

\end{enumerate}

\section*{Problem 2}
\enum
\item
\ba
L(\bbeta) & \propto p(\by | \lambda_i) \\
	& = \prod_{i=1}^n p(y_i | \lambda_i) \\
	& = \prod_{i=1}^n \lambda_i \e^{-\lambda_i y_i} \\
	& = \prod_{i=1}^n \lambda_i \exp\left(-\sum_{i=1}^n \lambda_i y_i \right) \\
	& = \prod_{i=1}^n \exp(\beta_0 + \beta_1 x_i)
		\exp\left(-\sum_{i=1}^n \exp(\beta_0 + \beta_1 x_i) y_i \right) \\
	& = \exp\left\{ \sum_{i=1}^n (\beta_0 + \beta_1 x_i) \right\}
		\exp\left(-\sum_{i=1}^n \exp(\beta_0 + \beta_1 x_i) y_i \right) \\
	& = \exp\left\{ \sum_{i=1}^n (\beta_0 + \beta_1 x_i)
			- \sum_{i=1}^n \exp(\beta_0 + \beta_1 x_i) y_i \right\} \\
	& = \boxed{ \exp\left\{ n\beta_0 + \sum_{i=1}^n \left[
			 \beta_1 x_i - \exp(\beta_0 + \beta_1 x_i) y_i \right]
		\right\} }
\ea
Thus, the log-likelihood is
\ba
\boxed{ 
\ell(\bbeta)  = n\beta_0 + \sum_{i=1}^n \left[
			 \beta_1 x_i - \exp(\beta_0 + \beta_1 x_i) y_i \right] }.
\ea
Next, we see that
\ba
{\pd{\ell}\over \pd{\beta_0}} = n 
	- \sum_{i=1}^n \exp(\beta_0 + \beta_1 x_i) y_i 
\ea
and
\ba
\pderiv{\ell}{\beta_1} & = \sum_{i=1}^n \left[ x_i -
	\exp(\beta_0 + \beta_1 x_i) y_i x_i \right] \\
	& = \sum_{i=1}^n x_i \left[1 - \exp(\beta_0 + \beta_1 x_i) y_i \right]
\ea
Hence, the score function is
\ba
\boxed{
\bS(\bbeta) = \deriv{\ell}{\bbeta}
	= \begin{pmatrix}
		n - \sum_{i=1}^n \exp(\beta_0 + \beta_1 x_i) y_i \\
		\sum_{i=1}^n x_i \left[1 - \exp(\beta_0 + \beta_1 x_i) y_i \right]
		\end{pmatrix}
		}
\ea
For the Fisher information matrix, we need to compute four more derivatives.
\ba
\pderiv{S_1}{\beta_0} =
	- \sum_{i=1}^n y_i \exp(\beta_0 + \beta_1 x_i),
	\htab
	\pderiv{S_1}{\beta_1} =
	- \sum_{i=1}^n y_i \exp(\beta_0 + \beta_1 x_i) \cdot x_i
\ea
\ba
\pderiv{S_2}{\beta_0} = 
	- \sum_{i=1}^n x_i y_i \exp(\beta_0 + \beta_1 x_i), 
	\htab 
	\pderiv{S_2}{\beta_1} =
		-\sum_{i=1}^n x_i y_i \exp(\beta_0 + \beta_1 x_i) \cdot x_i
\ea
Hence, the information matrix is 
\ba
\bI(\bbeta) & \equiv 
- \E{\pderiv{\bS}{\bbeta}} \\
	& = - \E{ \begin{pmatrix}
		- \sum_{i=1}^n Y_i \exp(\beta_0 + \beta_1 x_i)
		& - \sum_{i=1}^n x_i Y_i \exp(\beta_0 + \beta_1 x_i) \\
		- \sum_{i=1}^n x_i Y_i \exp(\beta_0 + \beta_1 x_i)
		& -\sum_{i=1}^n x_i^2 Y_i \exp(\beta_0 + \beta_1 x_i)
		\end{pmatrix}
		} \\
	& = \begin{pmatrix}
		\sum_{i=1}^n \exp(\beta_0 + \beta_1 x_i) \E{Y_i}
		& \sum_{i=1}^n x_i \exp(\beta_0 + \beta_1 x_i) \E{Y_i} \\
		\sum_{i=1}^n x_i \exp(\beta_0 + \beta_1 x_i) \E{Y_i}
		& \sum_{i=1}^n x_i^2 \exp(\beta_0 + \beta_1 x_i) \E{Y_i}
	\end{pmatrix} \\
	& = \begin{pmatrix}
		\sum_{i=1}^n \exp(\beta_0 + \beta_1 x_i)
		 \frac{1}{\exp(\beta_0 + \beta_1 x_i)}
		& \sum_{i=1}^n x_i \exp(\beta_0 + \beta_1 x_i) 
		\frac{1}{\exp(\beta_0 + \beta_1 x_i)} \\
		\sum_{i=1}^n x_i \exp(\beta_0 + \beta_1 x_i) 
		\frac{1}{\exp(\beta_0 + \beta_1 x_i)}
		& \sum_{i=1}^n x_i^2 \exp(\beta_0 + \beta_1 x_i) 
		\frac{1}{\exp(\beta_0 + \beta_1 x_i)}
	\end{pmatrix} \\
	& = \begin{pmatrix}
		n & \sum_{i=1}^n x_i \\
		\sum_{i=1}^n x_i & \sum_{i=1}^n x_i^2
		\end{pmatrix}
\ea
\item
The maximum likelihood estimate $\estim{\bbeta}$ is the value such that 
$\estim{\bbeta} = \argmax_{\bbeta} \ell(\bbeta)$, meaning that (for differentiable log-likelihood functions)
\ba
\pderiv{\ell}{\bbeta} \bigg|_{\bbeta= \estim{\bbeta}} 
= \begin{pmatrix} 0 \\ 0 \end{pmatrix},
\ea
that is,
\ba
\begin{pmatrix}
		n - \sum_{i=1}^n \exp(\estim{\beta_0} + \estim{\beta_1} x_i) y_i \\
		\sum_{i=1}^n x_i \left[1 - \exp(\estim{\beta_0} 
			+ \estim{\beta_1} x_i) y_i \right]
		\end{pmatrix}
		= \begin{pmatrix} 0 \\ 0 \end{pmatrix}.
\ea
Here, we cannot get an analytic form for $\estim{\beta_0}$ and $\estim{\beta_1}$, but we can implement a numerical solver for the implied system of equations above.
\item
The source code for this is in Appendix B.
\item

\enumend

\newpage

\lstinputlisting{Problem1.R}
\lstinputlisting{Problem2.R}

\end{document}



























