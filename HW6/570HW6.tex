\documentclass[11pt]{article}
\setlength{\topmargin}{-1in}
\setlength{\textheight}{10in}
\setlength{\oddsidemargin}{-0.25in}
\setlength{\textwidth}{7in}
\usepackage{mathpazo}
\usepackage[T1]{fontenc}
\usepackage[utf8]{inputenc}

\usepackage{amsmath}
\usepackage{amsthm}
\usepackage{amssymb}
\usepackage{appendix}
\usepackage{array}
\usepackage{bm}
\usepackage{cancel}
\usepackage{cite}
\usepackage{courier}
\usepackage{graphicx}
\usepackage{empheq}
\usepackage{enumitem}
\usepackage{listings}
\usepackage{mathtools}
\usepackage{units}
\usepackage{bigstrut}
\usepackage{rotating}
\usepackage{ mathrsfs }
\usepackage{multirow}
\usepackage{booktabs}
\usepackage{setspace}

\pagenumbering{gobble}

\usepackage{floatrow}
\floatsetup[figure]{capposition=top}
\DeclareMathAlphabet{\mathcal}{OMS}{cmsy}{m}{n}

\DeclareMathAlphabet{\mathsfit}{\encodingdefault}{\sfdefault}{m}{}
\SetMathAlphabet{\mathsfit}{bold}{\encodingdefault}{\sfdefault}{bx}{}

\newcommand{\tens}[1]{\bm{\mathsfit{#1}}}

\usepackage{color}
\lstset{language=R,basicstyle=\ttfamily,breaklines=true,
                keywordstyle=\color{blue}\ttfamily,
                stringstyle=\color{red}\ttfamily,
                commentstyle=\color{magenta}\ttfamily,
                showstringspaces=false,
                }

\newcommand*\widefbox[1]{\fbox{\hspace{2em}#1\hspace{2em}}}
\newcommand*\mb{\mathbf}
\newcommand*\reals{\mathbb{R}}
\newcommand*\complex{\mathbb{C}}
\newcommand*\naturals{\mathbb{N}}
\newcommand*\nats{\naturals}
\newcommand*\integers{\mathbb{Z}}
\newcommand*\rationals{\mathbb{Q}}
\newcommand*\irrationals{\mathbb{J}}
\newcommand*\pd{\partial}
\newcommand*\htab{\hspace{4 mm}}
\newcommand*\vtab{\vspace{0.5 in}}
\newcommand*\lsent{\mathcal{L}}
\newcommand*\mean[1]{\overline{#1}}
\newcommand*\union{\cup}
\newcommand*\intersect{\cap}
\newcommand*\cl{\cancel}
\newcommand*\ANS{\text{ANS}}
\newcommand*\As{\text{As}}
\newcommand*\then{\rightarrow}
\newcommand*\elim{\text{E}}
\newcommand*\intro{\text{I}}
\newcommand*\absurd{\curlywedge}
\newcommand*\NK{\vdash_{\text{NK}}}
\newcommand*\interp{\mathcal{I}}
\newcommand*\ba{\[ \begin{aligned}}
\newcommand*\ea{\end{aligned} \]}
\newcommand*\C{\mathcal{C}}
\newcommand*\card[1]{\text{card}\left(#1\right)}
\newcommand*\D{\mathscr{D}}
\newcommand*\df{=_{\text{def}}}
\newcommand*\eps{\epsilon}
\newcommand*\enum{\begin{enumerate}[label=(\alph*)]}
\newcommand*\enumend{\end{enumerate}}
\newcommand*\E[1]{\;\mathsf{E}\left[#1\right]}
\newcommand*\Esub[2]{\mathsf{E}_{#1}\left[#2\right]}
\newcommand*\Var[1]{\;\mathsf{Var}\left(#1\right)}
\newcommand*\SD[1]{\;\mathsf{SD}\left(#1\right)}
\newcommand*\SE[1]{\;\mathsf{SE}\left(#1\right)}
\newcommand*\Cov[1]{\;\mathsf{Cov}\left(#1\right)}
\newcommand*\iid{\overset{\text{iid}}{\sim}}
\newcommand*\Exp[1][\lambda]{\text{Exp}(\text{rate}=#1)}
\newcommand*\ind[1]{\mathbf{1}\left(#1\right)}
\newcommand*\set[1]{\left\{#1\right\}}
\newcommand*\estim[1]{\widehat{#1}}
\newcommand*\der{\text{d}}
\newcommand*\deriv[2]{\frac{\der #1}{\der #2}}
\newcommand*\pderiv[2]{\frac{\pd #1}{\pd #2}}
\newcommand*\norm[1]{\left\|#1\right\|}
\newcommand*\dist[2]{\;\text{dist}\left(#1, #2\right)}
\newcommand*\interior{\text{int}\;}
\newcommand*\exterior{\text{ext}\;}
\newcommand*\boundary{\text{bd}\;}
\newcommand*\lh{\overset{\text{L'H}}{=}}
\newcommand*\bA{\mathbf{A}}
\newcommand*\bb{\mathbf{b}}
\newcommand*\bB{\mathbf{B}}
\newcommand*\bc{\mathbf{c}}
\newcommand*\be{\mathbf{e}}
\newcommand*\bG{\mathbf{G}}
\newcommand*\bh{\mathbf{h}}
\newcommand*\bI{\mathbf{I}}
\newcommand*\bell{\boldsymbol{\ell}}
\newcommand*\bL{\mathbf{L}}
\renewcommand*\bm{\mathbf{m}}
\newcommand*\bo{\mathbf{o}}
\newcommand*\br{\mathbf{r}}
\newcommand*\bs{\mathbf{s}}
\newcommand*\bS{\mathbf{S}}
\newcommand*\bt{\mathbf{t}}
\newcommand*\bu{\mathbf{u}}
\newcommand*\bv{\mathbf{v}}
\newcommand*\bV{\mathbf{V}}
\newcommand*\bx{\mathbf{x}}
\newcommand*\bw{\mathbf{w}}
\newcommand*\bW{\mathbf{W}}
\newcommand*\bX{\mathbf{X}}
\newcommand*\by{\mathbf{y}}
\newcommand*\bY{\mathbf{Y}}
\newcommand*\bZ{\mathbf{Z}}
\newcommand*\bz{\mathbf{z}}
\newcommand*\bp{\mathbf{p}}
\newcommand*\bzero{\mathbf{0}}
\newcommand*\bone{\mathbf{1}}
\newcommand*\balpha{\boldsymbol{\alpha}}
\newcommand*\bbeta{\boldsymbol{\beta}}
\newcommand*\bgamma{\boldsymbol{\gamma}}
\newcommand*\beps{\boldsymbol{\varepsilon}}
\newcommand*\btheta{\boldsymbol{\theta}}
\newcommand*\bTheta{\boldsymbol{\Theta}}
\newcommand*\bmu{\boldsymbol{\mu}}
\newcommand*\bsigma{\boldsymbol{\sigma}}
\newcommand*\bSigma{\boldsymbol{\Sigma}}
\newcommand*\bOmega{\boldsymbol{\Omega}}
\newcommand\Psub[2]{\mathsf{P}_{#1}\left(#2\right)}
\newcommand\Vsub[2]{\mathsf{Var}_{#1}\left(#2\right)}
\newcommand\e{\operatorname{e}}
\newcommand\prox{\operatorname{prox}}
\newcommand\T{\mathsf{T}}

\newread\tmp
\newcommand\getcount{
	\openin\tmp=knot_count.txt
	\read\tmp to \knots
	\closein\tmp
	
	\openin\tmp=span.txt
	\read\tmp to \spanval
	\closein\tmp
	
	\openin\tmp=span_g.txt
	\read\tmp to \spantwo
	\closein\tmp
	
	\openin\tmp=span_g_2.txt
	\read\tmp to \spanthree
	\closein\tmp
}


\renewcommand\Re{\operatorname{Re}}
\renewcommand\Im{\operatorname{Im}}
\DeclareMathOperator*{\argmin}{arg\;min}
\DeclareMathOperator*{\argmax}{arg\;max}
\renewcommand\;{\,}
\renewcommand\epsilon{\varepsilon}
\renewcommand\rho{\varrho}
\renewcommand\phi{\varphi}
\renewcommand\mod{\hspace{0.2em} \textbf{mod}\hspace{0.2em}}
\renewcommand\Pr[1]{ \mathsf{Pr}\left(#1\right) }

\lstset{breaklines=true,
        numbersep=5pt,
        xleftmargin=.25in,
        xrightmargin=.25in}

\DeclareMathOperator{\sech}{sech}
\DeclareMathOperator{\sgn}{sgn}
\makeatletter
\renewcommand*\env@matrix[1][*\c@MaxMatrixCols c]{%
  \hskip -\arraycolsep
  \let\@ifnextchar\new@ifnextchar
  \array{#1}}
\makeatother

\newenvironment{amatrix}[1]{%
  \left(\begin{array}{@{}*{#1}{c}|c@{}}
}{%
  \end{array}\right)
}
\vspace{-1in}
\begin{document}
\title{STAT 570: Homework 6}
\author{Branden Olson}
\date{}
\maketitle

\section*{Problem 1}
\enum
\item
To explore the datasets, we first examine a joint plot of the two empirical cumulative distribution functions:
\begin{center}
\includegraphics[width=\linewidth]{cdfs.pdf}
\end{center}
Next, we look at a table of the quartiles, means, minima and maxima:
% latex table generated in R 3.4.1 by xtable 1.8-2 package
% Sat Nov 18 14:59:04 2017
\begin{table}[ht]
\centering
\begin{tabular}{rrrrrrl}
  \hline
Min. & 1st Qu. & Median & Mean & 3rd Qu. & Max. & Group \\ 
  \hline
171 & 396.8 & 681.5 & 823.2 & 1131 & 2415 & Hodgkin's \\ 
  116 & 360.0 & 433.0 & 522.0 & 709 & 1252 & Non-Hodgkin's \\ 
   \hline
\end{tabular}
\end{table}


From these two summaries, we do see evidence of a difference in the two distributions, so we should conduct formal analyses to determine if the difference is significant. 
\item
For (i), we consider the model
\begin{equation}
\E{Y_i} = \beta_0 + \beta_1 x_i,
\end{equation}
where as in (c),
\ba
x_i = \begin{cases}
		[1, 0], & i = 1, \dotsc, n \\
		[1, 1], & i = n + 1, \dotsc, 2n
	\end{cases}.
\ea
According to this model, $\beta_1$ represents the difference in means between the two datasets. Hence, a 90\% confidence interval for differences between distributions is a 90\% confidence interval for $\estim{\beta_1}$. Furthermore, we can see if this interval contains zero to determine whether we believe the difference to be significant or not. 

For (ii) and (iii), we consider the models 
\begin{equation}\label{logmodel}
\E{\ln{Y_i}} = \beta_0 + \beta_1 x_i
\end{equation}
and
\begin{equation}\label{sqrtmodel}
\E{\sqrt{Y_i}} = \beta_0 + \beta_1 x_i
\end{equation}
respectively. Similar reasoning yields that $\beta_1$ represents the difference in means between the two datasets on their transformed scale. Hence, we will also examine 90\% confidence intervals for $\estim{\beta_1}$ in \eqref{logmodel} and \eqref{sqrtmodel}. 
The table of confidence intervals for each model is shown below.
% latex table generated in R 3.4.1 by xtable 1.8-2 package
% Sat Nov 18 14:59:04 2017
\begin{table}[ht]
\centering
\begin{tabular}{rrl}
  \hline
5 \% & 95 \% & Model \\ 
  \hline
-541.549 & -60.751 & Original \\ 
  -0.756 & -0.041 & Log(Y) \\ 
  -9.532 & -0.881 & Sqrt(Y) \\ 
   \hline
\end{tabular}
\end{table}

In each case, we see that the CI does not contain zero, although it does come closer for the two transformed scales. When choosing a scale, we would like to meet the least squares assumptions as closely as possible since we are using \texttt{lm} for fitting. Perhaps the most straightforward way to assess each fit is to examine the residuals, shown below for each case. 
\begin{center}
\includegraphics[width=\linewidth]{resids.pdf}
\end{center}
The log transform seems to give the highest degree of heteroskedasticity.
For thoroughness we can also examine normal QQ-plots to assess normality of the residuals.
\begin{center}
    \includegraphics[width=\linewidth]{qqplots.pdf}
\end{center}
In each case there is good agreement, with some occassional skew that might be attributable to a small sample size.
We conclude that the log scale is likely the appropriate for analysis given these model assumption diagnostics.
\item
Here, we use similar modeling as in (b) but via the GLM framework. For a Poisson GLM, the canonical link is the logarithm, so that we have
\begin{equation}
\log\left(\E{Y_i}\right) = \beta_0 + \beta_1 x_i.
\end{equation}
For the Gamma distribution, the canonical link is the inverse function, so that
\begin{equation}
\frac{1}{\E{Y_i}} = \beta_0 + \beta_1 x_i
\end{equation}
Lastly, for the inverse Gaussian distribution, the canonical link is the inverse square, so that
\ba
\frac{1}{\left(\E{Y_i}\right)^2} = \beta_0 + \beta_1 x_i
\ea

For case (i),
\ba
\mu_i & = \exp\left(\beta_0 + \beta_1 x_i\right) \\
\mu_0 & = \e^{\beta_0} \\
\mu_1 & = \e^{\beta_0 + \beta_1}
\ea
Thus, equal means implies
\ba
\mu_1 - \mu_0 = 0 & \implies \e^{\beta_0 + \beta_1} - \e^{\beta_0} = 0 \\
    & \implies \beta_1 = 0.
\ea
Moreover, we have that
$f(\beta_0, \beta_1) := \e^{\beta_0 + \beta_1} - \e^{\beta_0}$ transforms the data to the original scale. Similarly, we see that taking $\beta_0' = \log(\beta_0)$ and $\beta_1' = \log(\beta_1)$ would equate the means and the sum of the $\beta_i$'s. 
\\
For case (ii),
\ba
\mu_i & = \frac{1}{\beta_0 + \beta_1 x_i} \\
\mu_0 & = \frac{1}{\beta_0} \\
\mu_1 & = \frac{1}{\beta_0 + \beta_1}
\ea
Thus, equal means would imply
\ba
\mu_1 - \mu_0 = 0 = \frac{1}{\beta_0 + \beta_1} - \frac{1}{\beta_0} \\
\implies \beta_1 = 0.
\ea
Moreover, we have that $f(\beta_0, \beta_1) = \frac{1}{\beta_0 + \beta_1} - \frac{1}{\beta_0}$ returns to the original scale.
Alternatively, we can take $\beta_0' = \frac{1}{\beta_0}$ and
\ba
\mu_1 = \frac{1}{\frac{1}{\beta_0} + \beta_1' } = \beta_1 \\
\implies \frac{1}{\beta_1} = \frac{1}{\beta_0} + \beta_1' \\
\implies \beta_1' = \frac{1}{\beta_1} - \frac{1}{\beta_0}. 
\ea
\\
For case (iii),
\ba
\mu_i & = \frac{1}{\sqrt{\beta_0 + \beta_1 x_i}} \\
\mu_0 & = \frac{1}{\sqrt{\beta_0}} \\
\mu_1 & = \frac{1}{\sqrt{\beta_0 + \beta_1}}
\ea
So that
\ba
\mu_1 - \mu_0 = 0 = \frac{1}{\sqrt{\beta_0 + \beta_1}} - \frac{1}{\sqrt{\beta_0}} \\
\implies \beta_1 = 0
\ea
Here, $f(\beta_0, \beta_1) = \frac{1}{\sqrt{\beta_0 + \beta_1}} - \frac{1}{\sqrt{\beta_0}}$ puts the problem on the original scale. Alternatively, set
\ba
\frac{1}{\sqrt{\beta_0'}} = \beta_0 \\
\frac{1}{\beta_0} = \sqrt{\beta_0'} \\
\beta_0' = \frac{1}{\beta_0^2}
\ea
and
\ba
\beta_1 = \frac{1}{\sqrt{ \frac{1}{\beta_0^2} + \beta_1' }} \\
\sqrt{\frac{1}{\beta_0^2} + \beta_1'} = \frac{1}{\beta_1} \\
\frac{1}{\beta_0^2} + \beta_1' = \frac{1}{\beta_1^2} \\
\beta_1' = \frac{1}{\beta_1^2} - \frac{1}{\beta_0^2}. 
\ea

\item
We use the \texttt{confint} function in R which automatically constructs the MLE-based confidence interval for a \texttt{glm} object.
A table displaying the 90\% CI for each model is given below.
% latex table generated in R 3.4.1 by xtable 1.8-2 package
% Sat Nov 18 14:59:04 2017
\begin{table}[ht]
\centering
\begin{tabular}{rrl}
  \hline
5 \% & 95 \% & Model \\ 
  \hline
-0.4760299 & -0.4348734 & Poisson \\ 
  0.0001920 & 0.0012454 & Gamma \\ 
  0.0000006 & 0.0000040 & Inverse Gaussian \\ 
   \hline
\end{tabular}
\end{table}

Under each model assumption, the confidence interval does not contain zero, and hence we would conclude that the means are not equal. 
Care needs to be taken, however, as a 95\% CI for the inverse Gaussian model does actually contain zero.

\enumend
\newpage
\section*{Appendix A: Source Code}
\lstinputlisting{P1.R}

\end{document}

















