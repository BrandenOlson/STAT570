\documentclass[11pt]{article}
\setlength{\topmargin}{-1in}
\setlength{\textheight}{10in}
\setlength{\oddsidemargin}{-0.25in}
\setlength{\textwidth}{7in}
\usepackage{mathpazo}
\usepackage[T1]{fontenc}
\usepackage[utf8]{inputenc}

\usepackage{amsmath}
\usepackage{amsthm}
\usepackage{amssymb}
\usepackage{appendix}
\usepackage{array}
\usepackage{bm}
\usepackage{cancel}
\usepackage{cite}
\usepackage{courier}
\usepackage{graphicx}
\usepackage{empheq}
\usepackage{enumitem}
\usepackage{listings}
\usepackage{mathtools}
\usepackage{units}
\usepackage{bigstrut}
\usepackage{rotating}
\usepackage{ mathrsfs }
\usepackage{multirow}
\usepackage{booktabs}
\usepackage{setspace}

\pagenumbering{gobble}

\usepackage{floatrow}
\floatsetup[figure]{capposition=top}
\DeclareMathAlphabet{\mathcal}{OMS}{cmsy}{m}{n}

\DeclareMathAlphabet{\mathsfit}{\encodingdefault}{\sfdefault}{m}{}
\SetMathAlphabet{\mathsfit}{bold}{\encodingdefault}{\sfdefault}{bx}{}

\newcommand{\tens}[1]{\bm{\mathsfit{#1}}}

\usepackage{color}
\lstset{language=R,basicstyle=\ttfamily,breaklines=true,
                keywordstyle=\color{blue}\ttfamily,
                stringstyle=\color{red}\ttfamily,
                commentstyle=\color{magenta}\ttfamily,
                showstringspaces=false,
                }

\newcommand*\widefbox[1]{\fbox{\hspace{2em}#1\hspace{2em}}}
\newcommand*\mb{\mathbf}
\newcommand*\reals{\mathbb{R}}
\newcommand*\complex{\mathbb{C}}
\newcommand*\naturals{\mathbb{N}}
\newcommand*\nats{\naturals}
\newcommand*\integers{\mathbb{Z}}
\newcommand*\rationals{\mathbb{Q}}
\newcommand*\irrationals{\mathbb{J}}
\newcommand*\pd{\partial}
\newcommand*\htab{\hspace{4 mm}}
\newcommand*\vtab{\vspace{0.5 in}}
\newcommand*\lsent{\mathcal{L}}
\newcommand*\conj{\overline}
\newcommand*\union{\cup}
\newcommand*\intersect{\cap}
\newcommand*\cl{\cancel}
\newcommand*\ANS{\text{ANS}}
\newcommand*\As{\text{As}}
\newcommand*\then{\rightarrow}
\newcommand*\elim{\text{E}}
\newcommand*\intro{\text{I}}
\newcommand*\absurd{\curlywedge}
\newcommand*\NK{\vdash_{\text{NK}}}
\newcommand*\derivation{\begin{tabular} { >{$}l<{$}  >{$}c<{$}  >{$}l<{$}  >{$}r<{$} }}
\newcommand*\interp{\mathcal{I}}
\newcommand*\ba{\[ \begin{aligned}}
\newcommand*\ea{\end{aligned} \]}
\newcommand*\C{\mathcal{C}}
\newcommand*\card[1]{\text{card}\left(#1\right)}
\newcommand*\D{\mathscr{D}}
\newcommand*\df{=_{\text{def}}}
\newcommand*\eps{\epsilon}
\newcommand*\enum{\begin{enumerate}[label=(\alph*)]}
\newcommand*\enumend{\end{enumerate}}
\newcommand*\E[1]{\mathsf{E}\left[#1\right]}
\newcommand*\Esub[2]{\mathsf{E}_{#1}\left[#2\right]}
\newcommand*\Var[1]{\text{Var}\left[#1\right]}
\newcommand*\Cov[1]{\;\text{Cov}\left[#1\right]}
\newcommand*\iid{\overset{\text{iid}}{\sim}}
\newcommand*\Exp[1][\lambda]{\text{Exp}(\text{rate}=#1)}
\newcommand*\ind[1]{\mathbf{1}\left(#1\right)}
\newcommand*\set[1]{\left\{#1\right\}}
\newcommand*\estim[1]{\widehat{#1}}
\newcommand*\der{\text{d}}
\newcommand*\deriv[2]{\frac{\der #1}{\der #2}}
\newcommand*\norm[1]{\left\|#1\right\|}
\newcommand*\dist[2]{\;\text{dist}\left(#1, #2\right)}
\newcommand*\interior{\text{int}\;}
\newcommand*\exterior{\text{ext}\;}
\newcommand*\boundary{\text{bd}\;}
\newcommand*\lh{\overset{\text{L'H}}{=}}
\newcommand*\bA{\mathbf{A}}
\newcommand*\bb{\mathbf{b}}
\newcommand*\bB{\mathbf{B}}
\newcommand*\bc{\mathbf{c}}
\newcommand*\be{\mathbf{e}}
\newcommand*\bh{\mathbf{h}}
\newcommand*\bI{\mathbf{I}}
\newcommand*\bL{\mathbf{L}}
\newcommand*\bo{\mathbf{o}}
\newcommand*\br{\mathbf{r}}
\newcommand*\bs{\mathbf{s}}
\newcommand*\bS{\mathbf{S}}
\newcommand*\bt{\mathbf{t}}
\newcommand*\bu{\mathbf{u}}
\newcommand*\bv{\mathbf{v}}
\newcommand*\bx{\mathbf{x}}
\newcommand*\bw{\mathbf{w}}
\newcommand*\bW{\mathbf{W}}
\newcommand*\bX{\mathbf{X}}
\newcommand*\by{\mathbf{y}}
\newcommand*\bY{\mathbf{Y}}
\newcommand*\bZ{\mathbf{Z}}
\newcommand*\bz{\mathbf{z}}
\newcommand*\bp{\mathbf{p}}
\newcommand*\bzero{\mathbf{0}}
\newcommand*\bone{\mathbf{1}}
\newcommand*\balpha{\boldsymbol{\alpha}}
\newcommand*\bbeta{\boldsymbol{\beta}}
\newcommand*\bgamma{\boldsymbol{\gamma}}
\newcommand*\beps{\boldsymbol{\varepsilon}}
\newcommand*\btheta{\boldsymbol{\theta}}
\newcommand*\bTheta{\boldsymbol{\Theta}}
\newcommand*\bmu{\boldsymbol{\mu}}
\newcommand*\bsigma{\boldsymbol{\sigma}}
\newcommand*\bSigma{\boldsymbol{\Sigma}}
\newcommand*\bOmega{\boldsymbol{\Omega}}
\newcommand\Psub[2]{\tens{P}_{#1}\left(#2\right)}
\newcommand\e{\operatorname{e}}
\newcommand\prox{\operatorname{prox}}
\newcommand\T{\mathsf{T}}

\newread\tmp
\newcommand\getcount{
	\openin\tmp=knot_count.txt
	\read\tmp to \knots
	\closein\tmp
	
	\openin\tmp=span.txt
	\read\tmp to \spanval
	\closein\tmp
	
	\openin\tmp=span_g.txt
	\read\tmp to \spantwo
	\closein\tmp
	
	\openin\tmp=span_g_2.txt
	\read\tmp to \spanthree
	\closein\tmp
}


\renewcommand\Re{\operatorname{Re}}
\renewcommand\Im{\operatorname{Im}}
\DeclareMathOperator*{\argmin}{arg\;min}
\DeclareMathOperator*{\argmax}{arg\;max}
\renewcommand\;{\,}
\renewcommand\epsilon{\varepsilon}
\renewcommand\rho{\varrho}
\renewcommand\phi{\varphi}
\renewcommand\mod{\hspace{0.2em} \textbf{mod}\hspace{0.2em}}
\renewcommand\Pr[1]{ \mathsf{Pr}\left(#1\right) }

\lstset{breaklines=true,
        numbersep=5pt,
        xleftmargin=.25in,
        xrightmargin=.25in}

\DeclareMathOperator{\sech}{sech}
\DeclareMathOperator{\sgn}{sgn}
\makeatletter
\renewcommand*\env@matrix[1][*\c@MaxMatrixCols c]{%
  \hskip -\arraycolsep
  \let\@ifnextchar\new@ifnextchar
  \array{#1}}
\makeatother

\newenvironment{amatrix}[1]{%
  \left(\begin{array}{@{}*{#1}{c}|c@{}}
}{%
  \end{array}\right)
}
\vspace{-1in}
\begin{document}
\title{STAT 570: Homework 4}
\author{Branden Olson}
\date{}
\maketitle

\section*{Problem 1}
\enum
\item
Our likelihood is of the form
\ba
L(\lambda| \by) & \propto p(\by| \lambda) \\
	& = \prod_{i=1}^n \left[\lambda \exp\left(-\lambda y_i\right)\right] \\
	& = \lambda^n \exp\left(-\lambda \sum_{i=1}^n y_i \right)
\ea
whereas our prior density is
\ba
\pi(\lambda) & = \frac{b^a}{\Gamma(a)} \lambda^{a - 1} \exp\left(-b \lambda\right).
\ea
Thus, our posterior is
\ba
p(\lambda | \by) & \propto L(\lambda | \by) \pi(\lambda) \\
	& = \lambda^n \exp\left(-\lambda \sum_{i=1}^n y_i \right)
		\frac{b^a}{\Gamma(a)} \lambda^{a - 1} \exp\left(-b \lambda\right) \\
	& \propto \lambda^{n + a - 1} 
		\exp\left(-\lambda\left[ b + \sum_{i=1}^n y_i \right] \right)
\ea
which is the kernel of a Gamma$\left(n + a, b + \sum_{i=1}^n y_i\right)$ density.
\item
Want
\ba
0.9 & = \Psub{\lambda}{ 0.1 \le \lambda \le 1} \\
	& = \Psub{\lambda}{ \lambda \le 1 } - \Psub{\lambda}{\lambda \le 0.1} \\
	& = F_\lambda(1) - F_\lambda(0.1)
\ea 
Note that $\lambda \sim \Gamma(a, b)$ implies
$\frac{2\lambda}{a} \sim \chi_{2b}^2$. Thus,
\ba
\Psub{\lambda}{\lambda \le c} & 
	= \Psub{\lambda}{ \frac{2\lambda}{a} < \frac{2c}{a} } \\
	& = \Pr{ \chi_{2b}^2 < \frac{2c}{a} } \\
	& = \chi_{2b, 2c/a}^2
\ea
which implies that
\ba
0.9 = \chi_{2b, 2\cdot 1/a}^2 - \chi_{2b, 2\cdot 0.1/a}^2
	= \chi_{2b, 2/a}^2 - \chi_{2b, 0.2/a}^2
\ea
\item
Below is the table of posterior means and standard deviations.
% latex table generated in R 3.4.1 by xtable 1.8-2 package
% Tue Oct 24 10:28:22 2017
\begin{table}[ht]
\centering
\begin{tabular}{rr}
  \hline
Mean$(\lambda | \by)$ & SD$(\lambda | \by)$ \\ 
  \hline
0.303 & 0.083 \\ 
  0.405 & 0.112 \\ 
  0.522 & 0.144 \\ 
  0.545 & 0.150 \\ 
   \hline
\end{tabular}
\end{table}


\item
First, we want
\ba
0.9 & = \Psub{\eta}{ 0.5 \le \eta \le 30 }  \\
	& = \Psub{\eta}{ 0.5 \le \text{LogNormal}(\mu_\eta, \sigma_\eta) \le 30 } \\
	& = \Psub{\eta}{ 0.5 \le \exp(\mu_\eta + \sigma_\eta Z) \le 30 },
		\htab Z \sim \mathcal{N}(0, 1) \\
	& = \Psub{\eta}{ \log(0.5) \le \mu_\eta + \sigma_\eta Z \le \log(30) } \\
	& = \Psub{\eta}{ \frac{\log(0.5) - \mu_\eta}{\sigma_\eta}
			\le \mathcal{N}(0, 1) \le
			\frac{\log(30) - \mu_\eta}{\sigma_\eta} } \\
	& = \Phi\left( \frac{\log(30) - \mu_\eta}{\sigma_\eta} \right)
		- \Phi\left( \frac{\log(0.5) - \mu_\eta}{\sigma_\eta} \right)
\ea



\enumend
 
\newpage
\section*{Appendix A: Problem 1 Source}
\lstinputlisting{P1.R}
\end{document}



























