\documentclass[11pt]{article}
\setlength{\topmargin}{-0.5in}
\setlength{\textheight}{9in}
\setlength{\oddsidemargin}{0.25in}
\setlength{\textwidth}{6in}
\usepackage{mathpazo}
\usepackage[T1]{fontenc}
\usepackage[utf8]{inputenc}

\usepackage{amsmath}
\usepackage{amsthm}
\usepackage{amssymb}
\usepackage{appendix}
\usepackage{array}
\usepackage{bm}
\usepackage{cancel}
\usepackage{cite}
\usepackage{courier}
\usepackage{graphicx}
\usepackage{empheq}
\usepackage{enumitem}
\usepackage{listings}
\usepackage{mathtools}
\usepackage{units}
\usepackage{bigstrut}
\usepackage{rotating}
\usepackage{ mathrsfs }
\usepackage{multirow}
\usepackage{booktabs}
\usepackage{setspace}

\pagenumbering{gobble}
\renewcommand{\baselinestretch}{1.5}

\usepackage{floatrow}
\floatsetup[figure]{capposition=top}
\DeclareMathAlphabet{\mathcal}{OMS}{cmsy}{m}{n}

\DeclareMathAlphabet{\mathsfit}{\encodingdefault}{\sfdefault}{m}{}
\SetMathAlphabet{\mathsfit}{bold}{\encodingdefault}{\sfdefault}{bx}{}

\newcommand{\tens}[1]{\bm{\mathsfit{#1}}}

\usepackage{color}
\lstset{language=R,basicstyle=\ttfamily,breaklines=true,
                keywordstyle=\color{blue}\ttfamily,
                stringstyle=\color{red}\ttfamily,
                commentstyle=\color{magenta}\ttfamily,
                showstringspaces=false,
                }

\newcommand*\widefbox[1]{\fbox{\hspace{2em}#1\hspace{2em}}}
\newcommand*\mb{\mathbf}
\newcommand*\reals{\mathbb{R}}
\newcommand*\complex{\mathbb{C}}
\newcommand*\naturals{\mathbb{N}}
\newcommand*\nats{\naturals}
\newcommand*\integers{\mathbb{Z}}
\newcommand*\rationals{\mathbb{Q}}
\newcommand*\irrationals{\mathbb{J}}
\newcommand*\pd{\partial}
\newcommand*\htab{\hspace{4 mm}}
\newcommand*\vtab{\vspace{0.5 in}}
\newcommand*\lsent{\mathcal{L}}
\newcommand*\conj{\overline}
\newcommand*\union{\cup}
\newcommand*\intersect{\cap}
\newcommand*\cl{\cancel}
\newcommand*\ANS{\text{ANS}}
\newcommand*\As{\text{As}}
\newcommand*\then{\rightarrow}
\newcommand*\elim{\text{E}}
\newcommand*\intro{\text{I}}
\newcommand*\absurd{\curlywedge}
\newcommand*\NK{\vdash_{\text{NK}}}
\newcommand*\derivation{\begin{tabular} { >{$}l<{$}  >{$}c<{$}  >{$}l<{$}  >{$}r<{$} }}
\newcommand*\interp{\mathcal{I}}
\newcommand*\ba{\[ \begin{aligned}}
\newcommand*\ea{\end{aligned} \]}
\newcommand*\C{\mathcal{C}}
\newcommand*\card[1]{\text{card}\left(#1\right)}
\newcommand*\D{\mathscr{D}}
\newcommand*\df{=_{\text{def}}}
\newcommand*\eps{\epsilon}
\newcommand*\enum{\begin{enumerate}[label=(\alph*)]}
\newcommand*\enumend{\end{enumerate}}
\newcommand*\E[1]{\mathsf{E}\left[#1\right]}
\newcommand*\Esub[2]{\mathsf{E}_{#1}\left[#2\right]}
\newcommand*\Var[1]{\text{Var}\left[#1\right]}
\newcommand*\Cov[1]{\;\text{Cov}\left[#1\right]}
\newcommand*\iid{\overset{\text{iid}}{\sim}}
\newcommand*\Exp[1][\lambda]{\text{Exp}(\text{rate}=#1)}
\newcommand*\ind[1]{\mathbf{1}\left(#1\right)}
\newcommand*\set[1]{\left\{#1\right\}}
\newcommand*\estim[1]{\widehat{#1}}
\newcommand*\der{\text{d}}
\newcommand*\deriv[2]{\frac{\der #1}{\der #2}}
\newcommand*\pderiv[2]{\frac{\pd #1}{\pd #2}}
\newcommand*\norm[1]{\left\|#1\right\|}
\newcommand*\dist[2]{\;\text{dist}\left(#1, #2\right)}
\newcommand*\interior{\text{int}\;}
\newcommand*\exterior{\text{ext}\;}
\newcommand*\boundary{\text{bd}\;}
\newcommand*\lh{\overset{\text{L'H}}{=}}
\newcommand*\bA{\mathbf{A}}
\newcommand*\bb{\mathbf{b}}
\newcommand*\bB{\mathbf{B}}
\newcommand*\bc{\mathbf{c}}
\newcommand*\be{\mathbf{e}}
\newcommand*\bh{\mathbf{h}}
\newcommand*\bI{\mathbf{I}}
\newcommand*\bK{\mathbf{K}}
\newcommand*\bL{\mathbf{L}}
\newcommand*\bo{\mathbf{o}}
\newcommand*\br{\mathbf{r}}
\newcommand*\bs{\mathbf{s}}
\newcommand*\bS{\mathbf{S}}
\newcommand*\bt{\mathbf{t}}
\newcommand*\bT{\mathbf{T}}
\newcommand*\bu{\mathbf{u}}
\newcommand*\bU{\mathbf{U}}
\newcommand*\bv{\mathbf{v}}
\newcommand*\bV{\mathbf{V}}
\newcommand*\bx{\mathbf{x}}
\newcommand*\bw{\mathbf{w}}
\newcommand*\bW{\mathbf{W}}
\newcommand*\bX{\mathbf{X}}
\newcommand*\by{\mathbf{y}}
\newcommand*\bY{\mathbf{Y}}
\newcommand*\bZ{\mathbf{Z}}
\newcommand*\bz{\mathbf{z}}
\newcommand*\bp{\mathbf{p}}
\newcommand*\bzero{\mathbf{0}}
\newcommand*\bone{\mathbf{1}}
\newcommand*\balpha{\boldsymbol{\alpha}}
\newcommand*\bbeta{\boldsymbol{\beta}}
\newcommand*\bgamma{\boldsymbol{\gamma}}
\newcommand*\beps{\boldsymbol{\varepsilon}}
\newcommand*\btheta{\boldsymbol{\theta}}
\newcommand*\bTheta{\boldsymbol{\Theta}}
\newcommand*\bmu{\boldsymbol{\mu}}
\newcommand*\bsigma{\boldsymbol{\sigma}}
\newcommand*\bSigma{\boldsymbol{\Sigma}}
\newcommand*\bOmega{\boldsymbol{\Omega}}
\newcommand\Psub[2]{\tens{P}_{#1}\left[#2\right]}
\newcommand\e{\operatorname{e}}
\newcommand\prox{\operatorname{prox}}
\newcommand\T{\mathsf{T}}

\newread\tmp
\newcommand\getcount{
	\openin\tmp=knot_count.txt
	\read\tmp to \knots
	\closein\tmp
	
	\openin\tmp=span.txt
	\read\tmp to \spanval
	\closein\tmp
	
	\openin\tmp=span_g.txt
	\read\tmp to \spantwo
	\closein\tmp
	
	\openin\tmp=span_g_2.txt
	\read\tmp to \spanthree
	\closein\tmp
}


\renewcommand\Re{\operatorname{Re}}
\renewcommand\Im{\operatorname{Im}}
\DeclareMathOperator*{\argmin}{arg\;min}
\DeclareMathOperator*{\argmax}{arg\;max}
\renewcommand\;{\,}
\renewcommand\epsilon{\varepsilon}
\renewcommand\rho{\varrho}
\renewcommand\phi{\varphi}
\renewcommand\mod{\hspace{0.2em} \textbf{mod}\hspace{0.2em}}
\renewcommand\Pr[1]{ \mathsf{Pr}\left[#1\right] }

\lstset{breaklines=true,
        numbersep=5pt,
        xleftmargin=.25in,
        xrightmargin=.25in}

\DeclareMathOperator{\sech}{sech}
\DeclareMathOperator{\sgn}{sgn}
\makeatletter
\renewcommand*\env@matrix[1][*\c@MaxMatrixCols c]{%
  \hskip -\arraycolsep
  \let\@ifnextchar\new@ifnextchar
  \array{#1}}
\makeatother

\newenvironment{amatrix}[1]{%
  \left(\begin{array}{@{}*{#1}{c}|c@{}}
}{%
  \end{array}\right)
}
\vspace{-1in}

\begin{document}
\title{STAT 570: Homework 2}
\author{Branden Olson}
\date{}
\maketitle

\section*{Problem 1}
\enum
\item
First, we show that $\Var{\eps_t} = \frac{\sigma^2}{1 - \phi^2}$ for all $t$ by induction. The base case is trivial, since we are given that 
$\eps_1 \sim \mathcal N\left(0, \frac{\sigma^2}{1 - \phi^2}\right)$. 
For the inductive step, assume that 
$\Var{\eps_{t-1}} = \frac{\sigma^2}{1 - \phi^2}$. Then we have, assuming independence of all $\eps_t$ and $\delta_t$,
\ba
\Var{\eps_t} & = \Var{\phi \eps_{t-1} + \delta_t} \\
	& = \phi^2 \Var{\eps_{t-1}} + \Var{\delta_t} \\
	& = \phi^2 \cdot \frac{\sigma^2}{1 - \phi^2} + \sigma^2 \\
	& = \frac{ \phi^2 \sigma^2 + \sigma^2(1 - \phi^2) }{1 - \phi^2} \\
	& = \frac{ \sigma^2 }{ 1 - \phi^2 }.
\ea
Thus, the result follows by mathematical induction (truncating, of course, at $t = n$). 
Next, we examine $\Cov{\eps_i, \eps_j}$ for $i \ne j$. Assume WLOG that $j > i$, since for example we can swap $j$ and $i$ by symmetry of covariance. We see that
\ba
\Cov{\eps_i, \eps_j} & = \E{ \eps_i \eps_j } - \E{\eps_i} \E{\eps_j } \\
	& = \E{\eps_i\left(\phi \eps_{j-1} + \delta_j\right) } - 0 \\
	& = \E{\eps_i\left(\phi\left(\phi \eps_{j - 2} 
		+ \delta_{j - 1} \right) + \delta_j \right) } \\
	& \htab \vdots \htab (j - i \;\;\; \text{times total}) \\
	& = \E{ \eps_i ( \phi( \dotsc (\phi \eps_i + \delta_i) + \dotsc + \delta_{j-1} ) + \delta_j } \\
	& = \E{ \phi^{j-i} \eps_i^2 + \phi^{j-i-1} \eps_i \delta_i +
		 \dotsc + \phi^2 \eps_i \delta_{j-2} 
		 + \phi \eps_i \delta_{j-1}} + \E{\delta_j} \\
	& = \phi^{j-i} \E{\eps_i^2} + \phi^{j-i-1} \E{\eps_i} \E{\delta_i}
		+ \dotsc + \phi^2 \E{\eps_i} \E{\delta_{j-2}} 
		+ \phi \E{\eps_i} \E{\delta_{j-1}} + 0
\ea
where we used that $\E{\eps_i \delta_k} = \E{\eps_i}\E{\delta_k}, \forall k$. But $\E{\eps_i} = 0 \; \forall i$ and $\E{\delta_k} = 0 \; \forall k$, so that
\ba
\Cov{\eps_i, \eps_j}
	& = \phi^{j-i} \left(\Var{\eps_i} + \E{\eps_i}^2\right) + 0 + \dotsc + 0 \\
	& = \phi^{j-i} \frac{\sigma^2}{1 - \phi^2}
\ea
$\forall j \ne i$, which, combined with the above result for $\Var{\eps_t}$ when $i = j$, yields exactly the form to be shown. 
\item
Computing the gradient directly (using a few formulae in Wakefield Appendix A),
\ba
\pderiv{}{\bbeta} \left[ (\by - \bx \bbeta)^T \bV(\phi)^{-1} (\by - \bx \bbeta) 
	\right]
	& = 2 \pderiv{}{\bbeta} \left[ (\by - \bx \bbeta)^T \right]
		\bV(\phi)^{-1} (\by - \bx \bbeta) \\
	& = 2 \pderiv{}{\bbeta} \left[ \by^T - \bbeta^T \bx^T \right]
		\bV(\phi)^{-1} (\by - \bx \bbeta) \\
	& = 2 \left( - \bI_p \bx^T \right) 
		\bV(\phi)^{-1} (\by - \bx \bbeta) \\
	& = -2 \bx^T \bV^{-1} \by + 2 \bx^T \bV^{-1} \bx \bbeta \\
	& \overset{\text{set}}{=} \bzero
\ea
Solving for $\bbeta$, we have
\ba
2 \bx^T \bV^{-1} \bx \bbeta = 2 \bx^T \bV^{-1} \by  \\
	\implies \boxed{ \estim\bbeta = \estim{\bbeta}_{\text{GLS}} =
		(\bx^T \bV^{-1}(\phi) \bx)^{-1} \bx^T \bV^{-1}(\phi) \by }.
\ea
Furthermore,
\ba
\Var{\estim{\bbeta}} & = 
	\Var{ (\bx^T \bV^{-1}(\phi) \bx)^{-1} \bx^T \bV^{-1}(\phi) \by } \\
	& = (\bx^T \bV^{-1}(\phi) \bx)^{-1} \bx^T \bV^{-1}(\phi) \Var{\by} 
		\left[ (\bx^T \bV^{-1}(\phi) \bx)^{-1} \bx^T \bV^{-1}(\phi) \right]^T \\
	& = (\bx^T \bV^{-1} \bx)^{-1} \bx^T \bV^{-1} \Var{\beps}
		\bV^{-T} \bx (\bx \bV^{-1} \bx)^{-T} \\
	& = (\bx^T \bV^{-1} \bx)^{-1} \bx^T \underbrace{\bV^{-1} \bV}_{\equiv \bI}
		 \bV^{-T} \bx (\bx \bV^{-1} \bx)^{-T} \\
	& = (\bx^T \bV^{-1} \bx)^{-1} \bx^T \bV^{-T}  
		 \bx (\bx \bV^{-1} \bx)^{-T} \\
	& = (\bx^T \bV^{-1} \bx)^{-1} 
		\underbrace{\left(\bx^T \bV^{-1} \bx\right)^T  
		 (\bx \bV^{-1} \bx)^{-T}}_{\equiv \bI} \\
	& = \left(\bx^T \bV^{-1} \bx\right)^{-1}
\ea
\item
We use \texttt{lm} to perform an OLS regression to estimate $\estim{\bbeta}$. Estimates along with their standard errors are shown below in part (d).

We also examine the autocorrelation plot of the OLS residuals:
\begin{center}
\includegraphics[width=\linewidth]{lm_acf.pdf}
\end{center}
Here, there is evidence of correlation between errors of lag length one, which matches the assumption of the model that $\eps_t$ is a function of $\eps_{t - 1}$.
The oscillatory pattern for larger lags suggests that there is dependence on more than just the first, and the magnitude of the dependence shrinks with increase in lag. 
\item
Below, the full table of OLS and GLS coefficients and standard errors is included, using $\phi=0.2$. All source code for this problem can be found in Appendix A.
% latex table generated in R 3.4.1 by xtable 1.8-2 package
% Fri Oct 20 11:47:16 2017
\begin{table}[ht]
\centering
\begin{tabular}{rrrr}
  \hline
$\estim{\bbeta}_{\text{OLS}}$ & SE($\estim{\bbeta}_{\text{OLS}}$) & $\estim{\bbeta}_{\text{GLS}}$ & SE($\estim{\bbeta}_{\text{GLS}}$) \\ 
  \hline
1097.75 & 24.13 & 1098.79 & 29.86 \\ 
  -247.78 & 28.44 & -249.53 & 35.14 \\ 
   \hline
\end{tabular}
\end{table}
 
We find high agreement between the estimates of $\beta_1$, which represents the estimated decline in the Nile volume from the original mean (estimated by $\beta_0$.  
\item
We aim to find the maximum likelihood estimator with respect to the joint pdf 
\\
$p(y_2, \dotsc, y_n | y_1, \bbeta, \sigma^2, \phi)$. 
Note that
\ba
y_t & = \beta_0 + \beta_1 x_t + \eps_t \\
	& = \beta_0 + \beta_1 x_t + \phi\eps_{t - 1} + \delta_t \\
	& = \beta_0 + \beta_1 x_t + 
		\phi\left(y_{t-1} - \beta_0 - \beta_1 x_{t-1}\right)
		+ \delta_t \\
	& = \mu_t + \phi(y_{t-1} - \mu_{t-1}) + \delta_t
\ea
so that $y_t$ only (directly) depends on the previous $y_{t - 1}$. By (a), and letting
 $\bmu = (\beta_0 + \beta_1 x_1, \dotsc, \beta_0 + \beta_1 x_n)^T$,  we have that
\ba
\by \sim \mathcal{N}_n \left( \bmu, \bV \right)
\ea
which means that the conditional marginals $y_t | y_{t-1}$ are also normal. Hence, it will suffice to find the conditional means and variances to specify these distributions. Indeed, we find
\ba
\E{y_t | y_{t_1}} 
	& = \E{ \mu_t + \phi(y_{t-1} - \mu_{t-1}) + \delta_t } \\
	& = \mu_t + \phi(y_{t-1} - \mu_{t-1})
\ea
and
\ba
\Var{y_t | y_{t_1}}
	& = \Var{ \mu_t + \phi(y_{t-1} - \mu_{t-1}) + \delta_t | y_{t-1}} \\
	& = \Var{\delta_t} \\
	& = \sigma^2
\ea
so that
\ba
y_t | y_{t-1} \sim \mathcal{N}\left( \mu_t + \phi(y_{t-1} - \mu_{t-1}), \sigma^2 \right)
\ea
Thus,
\ba
L(\phi | y_2, \dotsc, y_n, \bbeta, \sigma^2) 
	& = p(y_2, \dotsc, y_n | y_1, \bbeta, \sigma^2, \phi) \\
	& = \prod_{i=2}^n p(y_i | y_{i - 1}, \bbeta, \sigma^2) \\
	& = \prod_{i=2}^n \frac{1}{\sqrt{2\pi\sigma^2}}
		\exp\left( 
		- \frac{ \left[y_i - \mu_i - \phi(y_{i-1} - \mu_{i-1})\right]^2 }
						{ 2 \sigma^2 } \right) \\
	& \propto \exp\left\{-\frac{1}{2\sigma^2} \sum_{i=2}^{n} \left[
			y_i - \mu_i - \phi(y_{i-1} - \mu_{i - 1}) \right]^2
			\right\}
\ea
Hence,
\ba
\ell(\phi | y_2, \dotsc, y_n, \bbeta, \sigma^2)
	& = -\frac{1}{2\sigma^2} \sum_{i=2}^{n}
		 \left[
			y_i - \mu_i - \phi(y_{i-1} - \mu_{i - 1}) \right]^2
\ea
and
\ba
\pderiv{\ell}{\phi} & = 
	-\frac{1}{2\sigma^2} \sum_{i=2}^{n}
		2 \left[
			y_i - \mu_i - \phi(y_{i-1} - \mu_{i - 1}) \right]
		\cdot (\mu_{i-1} - y_{i-1})
		= 0 \\
	& \implies
	 \sum_{i=2}^{n} (y_i - \mu_i) (\mu_{i-1} - y_{i-1})
	 + \sum_{i=2}^{n} \phi(y_{i-1} - \mu_{i-1})^2 = 0 \\
	& \implies
	\phi \sum_{i=1}^{n-1} (y_i - \mu_i)^2 
		= \sum_{i=2}^n (y_i - \mu_i)(y_{i-1} - \mu_{i-1}) \\
	& \implies
	\estim{\phi} = \frac{ \sum_{i=2}^n (y_i - \mu_i)(y_{i-1} - \mu_{i-1}) }
		{ \sum_{i=1}^{n-1} (y_i - \mu_i)^2  }. \qed
\ea
Now, we have that
\ba
\pderiv{^2 \ell}{\phi^2} & =
	\frac{1}{\sigma^2} \sum_{i=2}^n (y_{i-1} - \mu_{i-1})(\mu_{i-1} - y_{i-1}) \\
	& = - \frac{1}{\sigma^2} \sum_{i=1}^{n-1} (y_i - \mu_i)^2
\ea
so that
\ba
I_{n-1}(\phi) 
	& =  \E{ - \pderiv{^2\ell}{\phi^2} } \\
	& =  \cdot \frac{1}{\sigma^2} \sum_{i=1}^{n-1} \E{ (Y_i - \mu_i)^2 } \\
	& = \frac{1}{\sigma^2} \sum_{i=1}^{n-1} \E{ \eps_i^2 } \\
	& = \frac{1}{\sigma^2} \sum_{i=1}^{n-1} 
		\left(\Var{\eps_i} + \E{\eps_i}^2 \right) \\
	& = \frac{1}{\sigma^2} \sum_{i=1}^{n-1} \left( \frac{\sigma^2}{1 - \phi^2}
			+ 0 \right) \htab \text{(by (a))} \\
	& = \frac{n - 1}{1 - \phi^2}.
\ea
Thus, since MLEs are CANE estimators, we have
\ba
\sqrt{\frac{n-1}{1 - \phi^2}}\left(\estim{\phi} - \phi\right)
	\to_d \mathcal N\left(0, \frac{1}{I_{n-1}(\phi)}\right) \\
	\implies
	\sqrt{n - 1} \left( \estim{\phi} - \phi\right)
	\to_d \mathcal N(0, 1 - \phi^2). \qed
\ea
\item
By (a), we have that $\by \sim \mathcal{N}_n(\mu, \bV)$, so that
\ba
L(\bbeta, \sigma^2, \phi | \by) 
	& \propto p(\by | \bbeta, \sigma^2, \phi) \\
	& = \left( 2 \pi \det(\bV) \right)^{-n/2}
		\exp\left\{ - \frac{(\by - \mu(\bbeta))^T \bV^{-1} 
		(\by - \mu(\bbeta))}{2}  \right\} 
\ea
yielding
\ba
\ell(\bbeta, \sigma^2, \phi | \by ) & = 
	- \frac{n}{2} \log\left( 2 \pi \det(\bV) \right)
	- \frac{(\by - \mu(\bbeta))^T \bV^{-1} 
		(\by - \mu(\bbeta))}{2}.
\ea
Using Wakefield appendix A,
\ba
S(\bbeta) =
\pderiv{\ell}{\bbeta} 
	& = - \frac{1}{2} \pderiv{}{\bbeta} \left[
		(\by - \mu(\bbeta))^T \bV^{-1} 
		(\by - \mu(\bbeta))
		\right] \\
	& = - \frac{1}{2}
		\cdot 2 \pderiv{(\by - \bX \bbeta)^T}{\bbeta} 
		\bV^{-1} (\by - \bX \bbeta) \\
	& = - \left( - \bX^T \right) V^{-1} \left( \by - \bX \bbeta \right) \\
	& = \boxed{ \bX^T \bV^{-1} \left(\by - \bX \bbeta\right) }
\ea

\ba
\pderiv{^2\ell}{\bbeta^2} =
	\bX^T \bV^{-1} (- \bX) = - \bX^T \bV^{-1} \bX
\ea
So that
\ba
I(\bbeta) = - \E{ \deriv{^2 \ell}{\bbeta^2} } = \bX^T \bV^{-1} \bX.
\ea
Hence, we have
\ba
\sqrt{n} \left[ \estim{\bbeta} - \bbeta \right]
	\to_d \mathcal{N} \left(0, I^{-1}(\bbeta) \right)
	= \mathcal{N} \left(0, \left(\bX^T \bV^{-1} \bX\right)^{-1} \right).
\ea
That is, the variance of the MLE converges to the variance of the GLS estimator found in (b).
\item
Note that the issue in computing the MLE of $\bbeta$ is that $\phi$ is also an unknown parameter that must be estimated. (Note that $\sigma^2$ does not actually need to be estimated, since it drops out of the GLS estimator.)
Hence, we can start by getting an OLS estimate of $\bbeta$, using this to estimate $\estim\phi$ by conditional likelihood via (e), and then impute this estimate to solve the $\bbeta$ score equation. More explicitly, we have:
\ba
\text{Step 1}: & \htab  \estim{\bbeta} \gets \left(\bX^T \bX\right)^{-1} \bX^T \bY \\
\text{Step 2}: & \htab \estim{\phi} | \estim{\bbeta} \gets \frac{ \sum_{i=2}^n (y_i - \mu_i(\estim{\bbeta})(y_{i-1} - \mu_{i-1}(\estim{\bbeta}) }
		{ \sum_{i=1}^{n-1} (y_i - \mu_i(\estim{\bbeta}))^2  } \\
\text{Step 3}: & \htab \estim{\bbeta} \gets \left(\bX^T \bV^{-1}(\estim{\phi}) \bX\right)^{-1}
	 \bX^T \bV^{-1}(\estim{\phi}) \bY \\
\text{Step 4}: & \htab \text{If } \norm{\estim{\bbeta}_n - \estim{\bbeta}_{n-1}} > \eps, \text{ goto Step 2}
\ea
where $\eps > 0$ is some error threshold.
\item
Below, we display the MLE estimate from our algorithm in (g). Our estimate of the decline, -249.09, is slightly larger than the GLS estimate found in (d), with a slightly smaller standard error. The decrease in SE is likely due to estimating $\phi$ rather than assuming its value.
% latex table generated in R 3.4.1 by xtable 1.8-2 package
% Fri Oct 20 11:47:16 2017
\begin{table}[ht]
\centering
\begin{tabular}{rr}
  \hline
$\estim{\beta}_\text{MLE}$ & SE($\estim{\beta}_\text{MLE}$) \\ 
  \hline
1098.53 & 28.55 \\ 
  -249.09 & 33.61 \\ 
   \hline
\end{tabular}
\end{table}

\enumend

\section*{Problem 2}
\enum
\item
\ba
h(y|\lambda)  = \frac{p(y|\lambda)}{\Pr{Y > y | \lambda}}  = \frac{ \lambda \e^{-\lambda y} }{ \e^{-\lambda y} }
\ea
Let $y_1, \dotsc, y_n \iid \text{Exponential}(\lambda)$. Then
\ba
L(\lambda | \by) & \propto p_{\bY}(\by | \lambda) \\
	& = \prod_{i=1}^n \lambda \e^{-\lambda y_i} \\
	& = \lambda^n \exp\left( - \lambda \sum_{i=1}^n y_i \right) \\
\ea
This yields the log-likelihood
\ba
\ell(\lambda | \by) = n \log(\lambda) - \lambda \sum_{i=1}^n y_i.
\ea
Taking the derivative and setting to zero,
\ba
\pderiv{\ell}{\lambda} = 
\frac{n}{\lambda} - \sum_{i=1}^n y_i = 0 
	& \implies \frac{n}{\lambda} = \sum_{i=1}^n y_i \\
	& \implies \estim\lambda = \estim\lambda_{\text{MLE}} = \frac{n}{\sum_{i=1}^n y_I} = \frac{1}{\overline{\by}}
\ea
Note that
\ba
\pderiv{^2 \ell}{\lambda^2} = - \frac{n}{\lambda^2}
\ea
so that
\ba
I(\lambda) = - \E{ \pderiv{^2 \ell}{\lambda^2} } = \frac{n}{\lambda^2}.
\ea
Hence, since MLEs are CANE for their parameter, we have
\ba
\Var{\estim{\lambda}} \to_d \frac{1}{nI(\lambda)} = \lambda^2.
\ea
That is,
\ba
\sqrt{n} \left( \estim{\lambda} - \lambda \right) \to_d \mathcal{N}(0, \lambda^2)
\ea
which suggests the standard error
\ba
\estim{\text{SE}} = \sqrt{ \frac{\estim{\lambda}^2}{n} } 
= \frac{\estim{\lambda}}{\sqrt n}.
\ea
\item
Below lists the estimates $\estim{\lambda}$ and their corresponding standard errors, for each length given in the dataset. 
% latex table generated in R 3.4.1 by xtable 1.8-2 package
% Fri Oct 20 09:37:56 2017
\begin{table}[ht]
\centering
\begin{tabular}{rr}
  \hline
$\estim{\lambda}$ & SE($\estim{\lambda}$) \\ 
  \hline
0.317 & 0.0879 \\ 
  0.433 & 0.1200 \\ 
  0.571 & 0.1584 \\ 
  0.600 & 0.1664 \\ 
   \hline
\end{tabular}
\end{table}

We examine our model assumptions by looking at an exponential QQ-plot for each value of length.
\begin{center}
\includegraphics[width=0.5\linewidth]{qqplots.pdf}
\end{center}
The QQ plots show that the data are very clearly not exponentially distributed, as there is left-skew in both tails. This matches what is seen in the histograms:
\begin{center}
\includegraphics[width=0.7\linewidth]{histograms.pdf}
\end{center}
\item
Here we see that 
\ba
\mu(\lambda) = \E{\bY | \lambda} = \frac{1}{\lambda}
\ea
and
\ba
\alpha \frac{1}{\lambda^2} & = \alpha \left(\frac{1}{\lambda}\right)^2
	= \alpha (\mu(\lambda))^2 = \alpha V(\mu(\lambda)) \\
	& \implies V(x) = x^2
\ea
An estimate of $\alpha$ is given by
\ba
\estim{\alpha}_n & := \frac{1}{n - 1}
	\sum_{i=1}^n \frac{ (Y_i - \estim{\mu}_i(\lambda))^2 }{ V(\estim{\mu}_i(\lambda)) } \\
	& = \frac{1}{n - 1}
	\sum_{i=1}^n \frac{ (Y_i - \frac{1}{\estim{\lambda}})^2 }
		{ \frac{1}{\estim{\lambda}^2} } \\
	& = \frac{1}{n - 1} \sum_{i=1}^n 
		\left( \estim{\lambda} Y_i - 1 \right)^2.
\ea
Now, our estimate $\estim{\lambda}$ solves the quasi-score equation
\ba
\bU_n(\lambda) & = \frac{1}{n} \sum_{i=1}^n \bU \left(\lambda, Y_i\right) \\
	& = \frac{1}{n} \sum_{i=1}^n \frac{D^T V^{-1} (Y_i - \mu_i(\lambda) )}{\alpha} \\
	& = \frac{1}{\alpha n} \sum_{i=1}^n \left(-\frac{1}{\lambda^2}\right) \lambda^2 \left(Y_i - \frac{1}{\lambda} \right) \\
	& = - \frac{1}{\alpha n} \sum_{i=1}^n \left(Y_i - \frac{1}{\lambda} \right) \\
	& = \frac{1}{\alpha} \left[ \frac{1}{\lambda} - \frac{1}{n} \sum_{i=1}^n Y_i \right] 
\ea
So that
\ba
\frac{1}{\estim \lambda} - \overline{\bY} = 0 \implies \estim{\lambda} = \frac{1}{\overline{\bY}}. 
\ea
This yields the asymptotic distribution given via
\ba
\left( \frac{\estim D^T \estim{V}^{-1} \estim D}{\estim{\alpha}_n} \right)^{1/2} 
	\left[ \estim \lambda - \lambda \right] 
	\to_d \mathcal N \left( 0, 1  \right) \\
	\implies 
\ea
where
\ba
D = \deriv{\mu}{\lambda} = \deriv{}{\lambda} \left[ \frac{1}{\lambda} \right]
 = - \frac{1}{\lambda^2}
\ea
and
\ba
\estim{V}^{-1} = \frac{1}{1/\estim{\lambda}^2} = \estim{\lambda}^2
\ea
which yields
\ba
\left( \frac{\frac{-1}{\estim{\lambda}^2} \estim{\lambda}^2 
	\frac{-1}{\estim{\lambda}^2}}{\estim{\alpha}_n} \right)^{1/2} \left[ \estim{\lambda} - \lambda \right] \to_d \mathcal{N} \left(0, 1 \right) \\
	\implies 
	\left( \frac{1}{\estim{\alpha}_n \estim{\lambda}^2} \right)^{1/2} 
		\left[ \estim{\lambda} - \lambda \right] \to_d 
		\mathcal N \left( 0, 1 \right) \\
	\implies
	\sqrt{\frac{n-1}{ \estim{\lambda}^2 \sum_{i=1}^n \left(\estim\lambda Y_i - 1\right)^2 } }
	\left[ \estim{\lambda} - \lambda \right] \to_d \mathcal N(0, 1)
\ea
Thus, 
\ba
\text{SE}\left(\estim{\lambda}\right) = 
	\sqrt{ \frac{ \estim{\lambda}^2 \sum_{i=1}^n \left(\estim\lambda Y_i - 1\right)^2 }{n - 1}}
\ea
% latex table generated in R 3.4.1 by xtable 1.8-2 package
% Fri Oct 20 09:37:56 2017
\begin{table}[ht]
\centering
\begin{tabular}{rrrr}
  \hline
Length & $\estim{\lambda}$ & $\estim{\alpha}$ & SE($\estim{\lambda}$) \\ 
  \hline
1 & 0.317 & 0.0184 & 0.0731 \\ 
  10 & 0.433 & 0.0055 & 0.0469 \\ 
  20 & 0.571 & 0.0231 & 0.1100 \\ 
  50 & 0.600 & 0.0105 & 0.0760 \\ 
   \hline
\end{tabular}
\end{table}
\\
The very low alphas strongly suggest that an exponential model is not the correct fit. 
\item
The empirical sandwich estimator for the variance is
\ba
\Var{\estim{\lambda}} = \frac{ \estim\bA^{-1} \estim\bB \estim\bA^{-T} }{n }
\ea
where
\ba
\bA = \bA_n = \frac{1}{n} \sum_{i=1}^n \pderiv{}{\lambda} G(\estim\lambda, Y_i)
\ea
and
\ba
\bB = \bB_n = \frac{1}{n} \sum_{i=1}^n G(\lambda, Y_i)^2
\ea
Using the \texttt{sandwich} package in R, we get the resultant table of SE estimates.
\begin{table}
\centering
\begin{tabular}{c|c|c|c|c}
Length & 1 & 10 & 20 & 50 \\
\hline
SE($\estim{\lambda}_\text{sandwich}$) & 0.00285 & 0.00153 & 0.00318 & 0.00209
\end{tabular}
\end{table}
\item
By definition we have
\ba
\E{Y} & = \int_0^\infty y \cdot \eta \alpha^{-\eta} y^{\eta - 1} 
	\exp\left[ -\left(\frac{y}{\alpha}\right)^\eta \right] \; \der y \\
	& = \frac{\eta}{\alpha^\eta} \int_0^\infty y^\eta 
		\exp\left[-\frac{y^\eta}{\alpha^\eta} \right] \; \der y.
\ea
Let $u = \frac{y^\eta}{\alpha^\eta}$, so that 
$\der u = \frac{\eta y^{\eta - 1}}{\alpha^\eta} \der y \implies \frac{\alpha^\eta}{\eta y^{\eta - 1}} \der u = \der y$.
Morever, this implies $y = (\alpha^\eta u)^{1/\eta} = \alpha u^{1/\eta}$, $u(0) = 0$, and $u(\infty) = \infty$. Thus,
\ba
\E{Y} & = \frac{\eta}{\alpha^\eta} \int_0^\infty y \cdot y^{\eta - 1} \e^{-u} \; \frac{\alpha^\eta}{\eta y^{\eta - 1}} \der u \\
	& = \int_0^\infty \alpha u^{1/\eta} \e^{-u} \; \der u \\
	& = \alpha \int_0^\infty u^{\left(\frac{1}{\eta} + 1\right) - 1} \e^{-u} \; \der u \\
	& = \boxed{ \alpha \Gamma\left(\frac{1}{\eta} + 1\right) }.
\ea
Furthermore, we have by a similar argument that
\ba
\E{Y^2} & = \eta \alpha^{-\eta} \int_0^\infty y^{n + 1} \exp\left(- \frac{y^\eta}{\alpha^\eta} \right) \; \der y \\
	& = \int_0^\infty y^2 \e^{-u} \; \der u \\
	& = \int_0^\infty (\alpha u^{1/\eta})^2 \e^{-u} \; \der u  \\
	& = \alpha^2 \int_0^\infty u^{\left(\frac{2}{\eta} + 1\right) - 1} \e^{-u} \; \der u \\
	& = \alpha^2 \Gamma\left( \frac{2}{\eta} + 1 \right) 
\ea
which yields
\ba
\Var{Y} & = \alpha^2 \Gamma\left(\frac{2}{\eta} + 1\right) - \left[ \alpha \Gamma\left(\frac{1}{\eta} + 1 \right) \right] \\
	& = \boxed{ \alpha^2 \left[ \Gamma\left(\frac{2}{\eta} + 1\right) - \Gamma^2\left(\frac{1}{\eta} + 1 \right) \right] }
\ea
As for the hazard function, we see that
\ba
S_Y(y | \eta, \alpha) & \equiv \Pr{ Y > y | \eta, \alpha } \\
	& = \int_y^\infty \eta \alpha^{-\eta} \xi^{\eta-1} \exp\left(- \frac{\xi^\eta}{\alpha^\eta} \right) \; \der u
\ea
Let $u = \frac{\xi^\eta}{\alpha^\eta}$, so that $\der u = \frac{\eta}{\alpha^\eta} \xi^{\eta - 1} \; \der \xi$. 
Moreover, $u(y) = \frac{y^\eta}{\alpha^\eta}$ and $u(\infty) = \infty$. Thus,
\ba
S_Y(y | \eta, \alpha) & = 
	\int_{y^\eta/\alpha^\eta}^\infty \e^{-u} \; \der u
	= \boxed{ \exp\left(- \left(\frac{y}{\alpha}\right)^\eta \right) }
\ea
Since $S(y) = \exp(-\lambda y)$ for an Exponential($\lambda$) random variable, we see that setting $\eta = 1$ and $\alpha = \frac{1}{\lambda}>0$ leads to an exponential hazard and thus an exponential distribution. 
\item
Note that
\ba
p(y | \eta, \alpha) & = \exp\left( \log(n) - \eta \log(\alpha) + (\eta - 1) \log(y) - \left(\frac{y}{\alpha}\right)^\eta \right) \\
	& = \exp\left( (\eta - 1) \log(y) - \frac{y^\eta}{\alpha^\eta} + \log(n) - \eta \log(\alpha) \right) \\
	& \ne \exp\left( \mathbf{a}(\eta, \alpha)^T \bT(\by) - A(\eta, \alpha) \right)
\ea
since, for example, we cannot multiplicatively decouple $\eta$ and $y$ in the factor $y^\eta$. Hence, the Weibull distribution is not a member of the exponential family. 
This implies a number of consequences for inference. Perhaps most important in this situation is that we are not guaranteed consistent estimators under model misspecification, e.g. in score-based estimation.
\item
Here, we have
\ba
L(\eta, \alpha | \by) & \propto p(\by | \eta, \alpha) \\
	& = \prod_{i=1}^n \eta \alpha^{-\eta} y_i^{\eta - 1} \exp\left[ - \left(\frac{y_i}{\alpha}\right)^\eta \right] \\
	& = \eta^n \alpha^{-n\eta} \left(\prod_{i=1}^n y_i \right)^{\eta-1} \exp\left[ - \alpha^{-\eta} \sum_{i=1}^n y_i^\eta \right]
\ea
and thus,
\ba
\boxed{ \ell(\eta, \alpha | \by) = n \log(\eta) - n \eta \log(\alpha) + (\eta - 1) \sum_{i=1}^n \log(y_i) - \sum_{i=1}^n \left(\frac{y_i}{\alpha}\right)^\eta }
\ea
Furthermore, 
\ba
\pderiv{\ell}{\eta} & = \frac{n}{\eta} - n \log(\alpha) + \sum_{i=1}^n \log(y_i) - \sum_{i=1}^n \left(\frac{y_i}{\alpha}\right)^\eta \log\left(\frac{y_i}{\alpha}\right) 
\ea
and
\ba
\pderiv{\ell}{\alpha} & = -\frac{n \eta}{\alpha} - \sum_{i=1}^n y_i^\eta \cdot (-\eta) \alpha^{-\eta - 1}
	= \eta \left( - \frac{n}{\alpha} + \sum_{i=1}^n \frac{y_i^\eta}{\alpha^{\eta + 1}} \right)
\ea
which yields the score
\ba
\boxed{
S(\eta, \alpha | \by) = \begin{pmatrix} 
	\frac{n}{\eta} - n \log(\alpha) + \sum_{i=1}^n \log(y_i) - \sum_{i=1}^n \left(\frac{y_i}{\alpha}\right)^\eta \log\left(\frac{y_i}{\alpha}\right) \\
	\eta \left( - \frac{n}{\alpha} + \sum_{i=1}^n \frac{y_i^\eta}{\alpha^{\eta + 1}} \right)
	\end{pmatrix}
	}
\ea
Further-furthermore, we have
\ba
\pderiv{^2\ell}{\eta^2} = -\frac{n}{\eta^2} - \sum_{i=1}^n \left(\frac{y_i}{\alpha}\right)^\eta \log^2 \left(\frac{y_i}{\alpha}\right)
\ea
\ba
\pderiv{^\ell}{\alpha^2} = \eta \left[ \frac{n}{\alpha^2} + \sum_{i=1}^n (-\eta - 1) \frac{y_i^\eta}{\alpha^{\eta + 2}} \right]
\ea
\ba
\pderiv{^\ell}{\alpha \pd \eta} & = \left( - \frac{n}{\alpha} + \sum_{i=1}^n \frac{y_i^\eta}{\alpha^{\eta + 1}} \right)
	+ \eta \left[ \sum_{i=1}^n \frac{1}{\alpha} \left(\frac{y_i}{\alpha}\right)^\eta \log\left(\frac{y_i}{\alpha}\right) \right] \\
		& = \sum_{i=1}^n \frac{y_i^\eta}{\alpha^{\eta + 1}} \left[ 1 + \eta \log\left(\frac{y_i}{\alpha}\right) \right]
			- \frac{n}{\alpha}
\ea
Hence,
\ba
\boxed{ 
I_n(\estim\eta, \estim\alpha) = - \begin{pmatrix}
	-\frac{n}{\estim\eta^2} - \sum_{i=1}^n \left(\frac{y_i}{\estim\alpha}\right)^{\estim\eta} \log^2 \left(\frac{y_i}{\estim\alpha}\right) &
	\sum_{i=1}^n \frac{y_i^{\estim\eta}}{\estim\alpha^{\estim\eta + 1}} \left[ 1 + \estim\eta \log\left(\frac{y_i}{\estim\alpha}\right) \right]
			- \frac{n}{\estim\alpha} \\
		\sum_{i=1}^n \frac{y_i^{\estim\eta}}{\estim\alpha^{\estim\eta + 1}} \left[ 1 + \estim\eta \log\left(\frac{y_i}{\estim\alpha}\right) \right]
			- \frac{n}{\estim\alpha} &
	\estim\eta \left[ \frac{n }{\estim\alpha^2} + \sum_{i=1}^n (-\estim\eta - 1) \frac{y_i^{\estim\eta}}{\estim\alpha^{\estim\eta + 2}} \right]
	\end{pmatrix}
	}
\ea
\item
Setting $S(\eta, \alpha | \by) = \bzero$ yields, in the second coordinate,
\ba
-\frac{n}{\alpha} + \sum_{i=1}^n \frac{y_i^\eta}{\alpha^{\eta + 1}} = 0 \\
	\implies \alpha^\eta n = \sum_{i=1}^n y_i^\eta \\
	\implies \alpha = \sqrt[\eta]{ \frac{1}{n} \sum_{i=1}^n y_i^\eta }.
\ea
Thus, we can plug this into the first coordinate to get 
\ba
\frac{n}{\eta} -
	 n \log\left(\sqrt[\eta]{ \frac{1}{n} \sum_{i=1}^n y_i^\eta }\right)
	  + \sum_{i=1}^n \log(y_i) - \sum_{i=1}^n \left(\frac{y_i}{\sqrt[\eta]{ \frac{1}{n} \sum_{i=1}^n y_i^\eta }}\right)^\eta \log\left(\frac{y_i}{\sqrt[\eta]{ \frac{1}{n} \sum_{i=1}^n y_i^\eta }}\right)
= 0
\ea
which is the only equation that needs to be solved numerically.
\item
Plugging the score equation above into \texttt{multiroot} in the \texttt{rootSolve} package yields the full table below of MLE values for $\eta$ and $\alpha$, along with their corresponding standard errors.

 latex table generated in R 3.4.1 by xtable 1.8-2 package
% Fri Oct 20 09:37:57 2017
\begin{table}[ht]
\centering
\begin{tabular}{rrrrr}
  \hline
Length & $\estim{\eta}$ & SE($\estim{\eta}$) & $\estim{\alpha}$ & SE($\estim{\alpha}$) \\ 
  \hline
1 & 10.33 & 2.35 & 3.32 & 0.0934 \\ 
  10 & 21.34 & 5.05 & 2.38 & 0.0322 \\ 
  20 & 9.64 & 2.31 & 1.85 & 0.0557 \\ 
  50 & 13.83 & 3.21 & 1.74 & 0.0365 \\ 
   \hline
\end{tabular}
\end{table}

\enumend

\newpage
\section*{Appendix A: Problem 1 Source}
\lstinputlisting{P1.R}

\newpage
\section*{Appendix B: Problem 2 Source}
\lstinputlisting{P2.R}
\end{document}



























