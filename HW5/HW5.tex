\documentclass[11pt]{article}
\setlength{\topmargin}{-1in}
\setlength{\textheight}{10in}
\setlength{\oddsidemargin}{-0.25in}
\setlength{\textwidth}{7in}
\usepackage{mathpazo}
\usepackage[T1]{fontenc}
\usepackage[utf8]{inputenc}

\usepackage{amsmath}
\usepackage{amsthm}
\usepackage{amssymb}
\usepackage{appendix}
\usepackage{array}
\usepackage{bm}
\usepackage{cancel}
\usepackage{cite}
\usepackage{courier}
\usepackage{graphicx}
\usepackage{empheq}
\usepackage{enumitem}
\usepackage{listings}
\usepackage{mathtools}
\usepackage{units}
\usepackage{bigstrut}
\usepackage{rotating}
\usepackage{ mathrsfs }
\usepackage{multirow}
\usepackage{booktabs}
\usepackage{setspace}

\pagenumbering{gobble}

\usepackage{floatrow}
\floatsetup[figure]{capposition=top}
\DeclareMathAlphabet{\mathcal}{OMS}{cmsy}{m}{n}

\DeclareMathAlphabet{\mathsfit}{\encodingdefault}{\sfdefault}{m}{}
\SetMathAlphabet{\mathsfit}{bold}{\encodingdefault}{\sfdefault}{bx}{}

\newcommand{\tens}[1]{\bm{\mathsfit{#1}}}

\usepackage{color}
\lstset{language=R,basicstyle=\ttfamily,breaklines=true,
                keywordstyle=\color{blue}\ttfamily,
                stringstyle=\color{red}\ttfamily,
                commentstyle=\color{magenta}\ttfamily,
                showstringspaces=false,
                }

\newcommand*\widefbox[1]{\fbox{\hspace{2em}#1\hspace{2em}}}
\newcommand*\mb{\mathbf}
\newcommand*\reals{\mathbb{R}}
\newcommand*\complex{\mathbb{C}}
\newcommand*\naturals{\mathbb{N}}
\newcommand*\nats{\naturals}
\newcommand*\integers{\mathbb{Z}}
\newcommand*\rationals{\mathbb{Q}}
\newcommand*\irrationals{\mathbb{J}}
\newcommand*\pd{\partial}
\newcommand*\htab{\hspace{4 mm}}
\newcommand*\vtab{\vspace{0.5 in}}
\newcommand*\lsent{\mathcal{L}}
\newcommand*\conj{\overline}
\newcommand*\union{\cup}
\newcommand*\intersect{\cap}
\newcommand*\cl{\cancel}
\newcommand*\ANS{\text{ANS}}
\newcommand*\As{\text{As}}
\newcommand*\then{\rightarrow}
\newcommand*\elim{\text{E}}
\newcommand*\intro{\text{I}}
\newcommand*\absurd{\curlywedge}
\newcommand*\NK{\vdash_{\text{NK}}}
\newcommand*\derivation{\begin{tabular} { >{$}l<{$}  >{$}c<{$}  >{$}l<{$}  >{$}r<{$} }}
\newcommand*\interp{\mathcal{I}}
\newcommand*\ba{\[ \begin{aligned}}
\newcommand*\ea{\end{aligned} \]}
\newcommand*\C{\mathcal{C}}
\newcommand*\card[1]{\text{card}\left(#1\right)}
\newcommand*\D{\mathscr{D}}
\newcommand*\df{=_{\text{def}}}
\newcommand*\eps{\epsilon}
\newcommand*\enum{\begin{enumerate}[label=(\alph*)]}
\newcommand*\enumend{\end{enumerate}}
\newcommand*\E[1]{\mathsf{E}\left[#1\right]}
\newcommand*\Esub[2]{\mathsf{E}_{#1}\left[#2\right]}
\newcommand*\Var[1]{\mathsf{Var}\left(#1\right)}
\newcommand*\SD[1]{\mathsf{SD}\left(#1\right)}
\newcommand*\Cov[1]{\;\text{Cov}\left[#1\right]}
\newcommand*\iid{\overset{\text{iid}}{\sim}}
\newcommand*\Exp[1][\lambda]{\text{Exp}(\text{rate}=#1)}
\newcommand*\ind[1]{\mathbf{1}\left(#1\right)}
\newcommand*\set[1]{\left\{#1\right\}}
\newcommand*\estim[1]{\widehat{#1}}
\newcommand*\der{\text{d}}
\newcommand*\deriv[2]{\frac{\der #1}{\der #2}}
\newcommand*\pderiv[2]{\frac{\pd #1}{\pd #2}}
\newcommand*\norm[1]{\left\|#1\right\|}
\newcommand*\dist[2]{\;\text{dist}\left(#1, #2\right)}
\newcommand*\interior{\text{int}\;}
\newcommand*\exterior{\text{ext}\;}
\newcommand*\boundary{\text{bd}\;}
\newcommand*\lh{\overset{\text{L'H}}{=}}
\newcommand*\bA{\mathbf{A}}
\newcommand*\bb{\mathbf{b}}
\newcommand*\bB{\mathbf{B}}
\newcommand*\bc{\mathbf{c}}
\newcommand*\be{\mathbf{e}}
\newcommand*\bh{\mathbf{h}}
\newcommand*\bI{\mathbf{I}}
\newcommand*\bL{\mathbf{L}}
\renewcommand*\bm{\mathbf{m}}
\newcommand*\bo{\mathbf{o}}
\newcommand*\br{\mathbf{r}}
\newcommand*\bs{\mathbf{s}}
\newcommand*\bS{\mathbf{S}}
\newcommand*\bt{\mathbf{t}}
\newcommand*\bu{\mathbf{u}}
\newcommand*\bv{\mathbf{v}}
\newcommand*\bV{\mathbf{V}}
\newcommand*\bx{\mathbf{x}}
\newcommand*\bw{\mathbf{w}}
\newcommand*\bW{\mathbf{W}}
\newcommand*\bX{\mathbf{X}}
\newcommand*\by{\mathbf{y}}
\newcommand*\bY{\mathbf{Y}}
\newcommand*\bZ{\mathbf{Z}}
\newcommand*\bz{\mathbf{z}}
\newcommand*\bp{\mathbf{p}}
\newcommand*\bzero{\mathbf{0}}
\newcommand*\bone{\mathbf{1}}
\newcommand*\balpha{\boldsymbol{\alpha}}
\newcommand*\bbeta{\boldsymbol{\beta}}
\newcommand*\bgamma{\boldsymbol{\gamma}}
\newcommand*\beps{\boldsymbol{\varepsilon}}
\newcommand*\btheta{\boldsymbol{\theta}}
\newcommand*\bTheta{\boldsymbol{\Theta}}
\newcommand*\bmu{\boldsymbol{\mu}}
\newcommand*\bsigma{\boldsymbol{\sigma}}
\newcommand*\bSigma{\boldsymbol{\Sigma}}
\newcommand*\bOmega{\boldsymbol{\Omega}}
\newcommand\Psub[2]{\mathsf{P}_{#1}\left(#2\right)}
\newcommand\Vsub[2]{\mathsf{Var}_{#1}\left(#2\right)}
\newcommand\e{\operatorname{e}}
\newcommand\prox{\operatorname{prox}}
\newcommand\T{\mathsf{T}}

\newread\tmp
\newcommand\getcount{
	\openin\tmp=knot_count.txt
	\read\tmp to \knots
	\closein\tmp
	
	\openin\tmp=span.txt
	\read\tmp to \spanval
	\closein\tmp
	
	\openin\tmp=span_g.txt
	\read\tmp to \spantwo
	\closein\tmp
	
	\openin\tmp=span_g_2.txt
	\read\tmp to \spanthree
	\closein\tmp
}


\renewcommand\Re{\operatorname{Re}}
\renewcommand\Im{\operatorname{Im}}
\DeclareMathOperator*{\argmin}{arg\;min}
\DeclareMathOperator*{\argmax}{arg\;max}
\renewcommand\;{\,}
\renewcommand\epsilon{\varepsilon}
\renewcommand\rho{\varrho}
\renewcommand\phi{\varphi}
\renewcommand\mod{\hspace{0.2em} \textbf{mod}\hspace{0.2em}}
\renewcommand\Pr[1]{ \mathsf{Pr}\left(#1\right) }

\lstset{breaklines=true,
        numbersep=5pt,
        xleftmargin=.25in,
        xrightmargin=.25in}

\DeclareMathOperator{\sech}{sech}
\DeclareMathOperator{\sgn}{sgn}
\makeatletter
\renewcommand*\env@matrix[1][*\c@MaxMatrixCols c]{%
  \hskip -\arraycolsep
  \let\@ifnextchar\new@ifnextchar
  \array{#1}}
\makeatother

\newenvironment{amatrix}[1]{%
  \left(\begin{array}{@{}*{#1}{c}|c@{}}
}{%
  \end{array}\right)
}
\vspace{-1in}
\begin{document}
\title{STAT 570: Homework 4}
\author{Branden Olson}
\date{}
\maketitle

\section*{Problem 1}
\enum
\item
We have
\ba
\theta & = 
	\frac{ \frac{ \Pr{Y = 1 | X = 1} }{\Pr{Y = 0| X = 1} } }
	{ \frac{\Pr{Y = 1 | X = 0}}{ \Pr{ Y = 0 | X = 0} } } \\
	& = \frac{ \Pr{Y = 1 | X = 1} }{\Pr{Y = 0| X = 1} }
		\cdot \frac{ \Pr{ Y = 0 | X = 0} }{\Pr{Y = 1 | X = 0}} \\
	& = \frac{ \Pr{X = 1 | Y = 1} \cdot \frac{\Pr{Y = 1}}{\Pr{X = 1}} }
		{\Pr{X = 1 | Y = 0} \cdot \frac{\Pr{Y = 0}}{\Pr{X = 1}}}
		\cdot
		\frac{ \Pr{X = 0| Y = 0} \cdot \frac{\Pr{Y = 0}}{\Pr{X = 0}} }
		{ \Pr{X = 0| Y = 1} \cdot \frac{\Pr{Y = 1}}{\Pr{X = 0}} } 
		\htab \text{(Bayes thm applied 4$\times$)} \\
	& = \frac{ \Pr{X = 1 | Y = 1} \cdot \Pr{Y = 1} }
		{\Pr{X = 1 | Y = 0} \cdot \Pr{Y = 0} }
		\cdot
		\frac{ \Pr{X = 0| Y = 0} \cdot \Pr{Y = 0} }
		{ \Pr{X = 0| Y = 1} \cdot\Pr{Y = 1} } \\
	& = \frac{ \Pr{X = 1 | Y = 1}  }
		{\Pr{X = 1 | Y = 0} }
		\cdot
		\frac{ \Pr{X = 0| Y = 0}  }
		{ \Pr{X = 0| Y = 1}  } \\
	& = \frac{ \Pr{X = 1 | Y = 1}  / \Pr{ X = 0 | Y = 1 } }
		{\Pr{X = 1 | Y = 0} / \Pr{ X = 0 | Y = 0 } } \\
	& = \frac{ p_1 /(1 - p_1) }{ p_2 / (1 - p_2) } \qed
\ea
\item
Since MLEs are invariant under transformations, we can find 
$\estim{p_1}_\text{MLE}$ and $\estim{p_2}_\text{MLE}$.
Indeed, a standard result tells us that
\ba
\sqrt{n}\left( \begin{pmatrix} \estim{p_1} \\ \estim{p_2} \end{pmatrix}
	 - \begin{pmatrix} p_1 \\ p_2 \end{pmatrix} \right)
	\to_d 
	\mathcal{N} \left( \bzero,
		\begin{pmatrix} p_1(1 - p_1) & 0 \\
		0 & p_2(1 - p_2)
		\end{pmatrix} \right).
\ea
with
\ba
\estim{p_i} = \frac{ X_i }{ n_i }
\ea
Let $g(p_1, p_2) = \frac{p_1/(1 - p_1)}{p_2/(1 - p_2)} \equiv \theta$. 
Then
\ba
\pderiv{g}{p_1} & = \frac{1 - p_2}{p_2} 
	\cdot \frac{(1-p_1) - p_1(-1)}{(1 - p_1)^2} \\
	& = \frac{1 - p_2}{p_2} \frac{1}{(1 - p_1)^2}.
\ea
\ba
\pderiv{g}{p_2} & = \frac{p_1}{1 - p_1} \cdot
	\frac{ p_2 (-1) - (1 - p_2) }{ p_2^2 } \\
	& = \frac{ -p_1}{(1 - p_1) p_2^2 } .
\ea

Hence, assuming that $\frac{n_1}{n_1 + n_2} \to c \in (0, 1)$,
\ba 
\sqrt{n_1 + n_2} \left(g(\estim{p_1}, \estim{p_2}) - g(p_1, p_2)\right)
	\to \mathcal N \left( 0, \nabla g(p_1, p_2)^T 
	\Sigma \nabla g(p_1, p_2) \right)
\ea
where
\ba
\nabla g(p_1, p_2)^T 
	\Sigma \nabla g(p_1, p_2)
	& = \left( \frac{1 - p_2}{p_2(1 - p_1)^2}, -\frac{p_1}{p_2^2(1 - p_1)} \right) 
		\begin{pmatrix} p_1(1 - p_1) & 0 \\
		0 & p_2(1 - p_2)
		\end{pmatrix}
		\begin{pmatrix} \frac{1 - p_2}{p_2(1 - p_1)^2} \\ -\frac{p_1}{p_2^2(1 - p_1)}  \end{pmatrix} \\
	& = \left( \frac{p_1(1 - p_2)}{p_2 (1 - p_1)}, - \frac{p_1 (1 - p_2)}{ p2(1 - p_1) } \right)
		\begin{pmatrix} \frac{1 - p_2}{p_2(1 - p_1)^2} \\ -\frac{p_1}{p_2^2(1 - p_1)}  \end{pmatrix} \\
	& = \frac{ p_1 (1 - p_2)^2 }{ p_2^2 (1 - p_1)^3} + \frac{ p_1^2 ( 1 - p_2) }{p_2^3 (1 - p_1)^2 } \\
	& = \frac{ p_1 p_2 (1 - p_2)^2 + p_1^2 (1 - p_1)(1 - p_2) }{ p_2^3 (1 - p_1)^3 } \\
	& = \frac{ p_1(1 - p_2)\left[ p_2(1 - p_2) + p_1(1 - p_1) \right] }{ p_2^3 (1 - p_1)^3 } \\
\ea
That is, we have
\ba
\sqrt{n_1 + n_2} \left( \estim{\theta} - \theta \right) \to_d \mathcal{N}\left(0, 
	\frac{ p_1(1 - p_2)\left[ p_2(1 - p_2) + p_1(1 - p_1) \right] }{ p_2^3 (1 - p_1)^3 } \right)
\ea
which yields the following estimate of standard error:
\ba
\text{SE}\left(\estim{\theta}\right) = 
 \sqrt{ \frac{ \estim{p_1}(1 - \estim{p_2})\left[ \estim{p_2}(1 - \estim{p_2}) 
 	+ \estim{p_1}(1 - \estim{p_1}) \right] }
	{ (n_1 + n_2) \estim{p_2}^3 (1 - \estim{p_1})^3 } }
\ea
Hence, a confidence interval is
\ba
\estim{\theta} \in \estim{\theta} \pm z_{1 - 0.1/2} \text{SE}\left(\estim{\theta}\right)
	= (4.60, 6.68).
\ea
\item
Using Bayes' formula, we have
\ba
p(p_i | x_i) & \propto
	L(x_i | p_i) \pi(p_i) \\
	& \propto {n_i \choose x_i} p_i^{x_i} (1 - p_i)^{n_i - x_i}
		\cdot p_i^{a - 1} (1 - p_i)^{b - 1} \\
	& \propto p_i^{x_i + a - 1} (1 - p_i)^{n_i - x_i + b - 1}
\ea
which is exactly the kernel of a Beta($a + x_i$, $b + n_i - x_i$). 
\item
Note that for $X \sim \text{Beta}(\alpha, \beta)$, we have
\ba
\E{X} = \frac{\alpha}{\alpha + \beta},
\ea
\ba
\text{Mode}(X) = \frac{\alpha - 1}{\alpha + \beta - 2}
\ea
and
\ba
\Var{X} = \frac{ \alpha \beta }{(\alpha + \beta)^2 (\alpha + \beta + 1) } \\
\implies \SD{X} = \frac{\sqrt{\alpha \beta}}
	{(\alpha + \beta) \sqrt{\alpha + \beta + 1} }
\ea
If $a = b = 1$, $p_i | x_i \sim \text{Beta}(1 + x_i, 1 + n_i - x_i)$, and we have
\ba
\E{p_i | X_i} & = \frac{1 + X_i}{1 + X_i + 1 + n_i - X_i} \\
	& = \frac{X_i + 1}{ n_i + 2 },
\ea
\ba
\text{Mode}(p_i | X_i) = \frac{X_i}{ n_i + 2 - 2} = \frac{X_i}{n_i}
\ea
and
\ba
\SD{p_i | X_i} & = \frac{\sqrt{(1 + x_i)(1 + n_i - x_i)}}
	{ (2 + n_i)\sqrt{2 + n_i + 1} } \\
	& = \frac{\sqrt{(X_i + 1)(n_i - X_i + 1)}}
		{(n_i + 2)\sqrt{n_i + 3}}.
\ea
These summaries for the case and control groups, as well as 90\% credible intervals given by 5\% and 95\% posterior quantiles, are shown in the table below.
% latex table generated in R 3.4.1 by xtable 1.8-2 package
% Thu Nov  2 13:10:20 2017
\begin{table}[ht]
\centering
\begin{tabular}{rrrrrl}
  \hline
Mean & Mode & SD & 0.05 Quantile & 0.95 Quantile & Group \\ 
  \hline
0.480 & 0.480 & 0.035 & 0.423 & 0.538 & Case \\ 
  0.142 & 0.141 & 0.012 & 0.122 & 0.163 & Control \\ 
   \hline
\end{tabular}
\end{table}

\item
We have that
\ba
\left( \estim{p_i}_\text{Bayes} | X_i \right) & \to_d \mathcal N \left( \E{ \estim{p_i}_\text{MLE} | X_i },
 \Var{ \estim{p_i}_\text{MLE} | X_i } \right) \\
	& = \mathcal N \left( p_i, p_i (1 - p_i) \right)
\ea
so that an approximate credible interval is given by the 0.05 and 0.95 quantiles of a $\mathcal{N}\left(\estim{p_i}, \frac{\estim{p_i}(1 - \estim{p_i})}{\sqrt{n_i}}\right)$,
where $\estim{p_i} = X_i/n_i$ for $i = 1, 2$.
The numerical approximations are shown below.
% latex table generated in R 3.4.1 by xtable 1.8-2 package
% Thu Nov  2 13:10:20 2017
\begin{table}[ht]
\centering
\begin{tabular}{rrl}
  \hline
0.05 Quantile & 0.95 Quantile & Group \\ 
  \hline
0.451 & 0.509 & Case \\ 
  0.134 & 0.148 & Control \\ 
   \hline
\end{tabular}
\end{table}

We see that they match quite well with the ones given from the finite-sample Beta quantiles above.
\item
\begin{center}
\includegraphics[width=\linewidth]{hist_case.pdf}
\end{center}

\begin{center}
\includegraphics[width=\linewidth]{hist_control.pdf}
\end{center}
The table of sample quantiles is shown below:
% latex table generated in R 3.4.1 by xtable 1.8-2 package
% Thu Nov  2 13:10:20 2017
\begin{table}[ht]
\centering
\begin{tabular}{rrl}
  \hline
5\% & 95\% & Group \\ 
  \hline
0.423 & 0.535 & Case \\ 
  0.122 & 0.162 & Control \\ 
   \hline
\end{tabular}
\end{table}

\item
Here we compute the samples and histogram based on our samples of $p_1$ and $p_2$ in part (f). 

\begin{center}
\includegraphics[width=\linewidth]{theta.pdf}
\end{center}
\newpage
Moreover, the corresponding sample quantiles are shown below.
% latex table generated in R 3.4.1 by xtable 1.8-2 package
% Thu Nov  2 13:10:20 2017
\begin{table}[ht]
\centering
\begin{tabular}{rrr}
  \hline
5\% & 50\% & 95\% \\ 
  \hline
4.211 & 5.593 & 7.630 \\ 
   \hline
\end{tabular}
\end{table}

\item
The given rate corresponds to $\Pr{Y = 1}$. Thus, by Bayes' theorem,
\ba
q_1 & = \Pr{Y = 1 | X = 1} \\
	& = \Pr{X = 1 | Y = 1} \frac{\Pr{Y = 1}}{\Pr{X = 1}} \\
	& = \frac{ \Pr{X = 1|Y = 1} \Pr{Y = 1} }
		{ \Pr{X = 1|Y = 1}\Pr{Y = 1} + \Pr{X = 1|Y = 0}\Pr{Y = 0} } \\
	& \approx \frac{ \estim{p_1} \frac{17}{100,000} }
		{ \estim{p_1} \frac{17}{100,000} 
			+ \estim{p_2} \left[ 1 - \frac{17}{100,000} \right] }
\ea
Similarly, we have
\ba
q_2 & = \Pr{Y = 1 | X = 0} \\
	& = \frac{ \Pr{X = 0 | Y = 1} \Pr{Y = 1} }{ \Pr{X = 0} } \\
	& \approx 
		\frac{(1 - \estim{p_1}) \frac{17}{100,000} }
		{ (1 - \estim{p_1}) \frac{17}{100,000}
			+ (1 - \estim{p_2}) \left[1-\frac{17}{100,000}\right] }.
\ea
\enumend

\section*{Problem 2}
By Bayes' theorem, we have
\ba
p(\theta | \estim{\theta}) & \propto
	\exp\left( - \frac{ (\theta - \estim{\theta})^2 }{ V } \right)
	\exp\left(-\frac{\theta^2}{W}\right) \\
	& = \exp\left( -\frac{\theta^2 - 2\theta \estim{\theta} 
		+ \estim{\theta}^2}{V} - \frac{\theta^2}{W} \right) \\
	& = \exp\left( - \left[ 
		\theta^2\left(\frac{1}{V} + \frac{1}{W}\right) 
		- \frac{2\estim{\theta}}{V} \theta 
		+ \frac{\estim{\theta}^2}{V}
		\right] \right) \\
	& = \exp\left( 
		- \frac{ \theta^2 
		- \frac{2\estim{\theta}}{V\left(\frac{V + W}{VW}\right)} \theta
		+ \frac{\estim{\theta}^2}{V\left(\frac{V + W}{VW}\right)} }
		{ \frac{VW}{V + W} } \right) \\
	& = \exp\left( 
		- \frac{ \theta^2 
		- \frac{2\estim{\theta}W}{V + W} \theta
		+ \frac{\estim{\theta}^2W}{V + W} 
		+ \left(\frac{\estim{\theta} W}{V + W}\right)^2
		- \left(\frac{\estim{\theta} W}{V + W}\right)^2 
		}
		{ \frac{VW}{V + W} } \right) \\
	& \propto \exp\left(
		- \frac{ \left(\theta - \frac{\estim{\theta} W}{ V + W}\right)^2 }
		{ \frac{VW}{V + W } } \right)
\ea
which is the kernel of a $\mathcal N \left(\frac{\estim{\theta}W}{V + W},
\frac{VW}{V + W}\right) = \mathcal N\left( r\estim{\theta}, rV\right)$. \qed

\section*{Problem 3}
\enum
\item
Noting that $Y_i \sim \mathcal N(\beta_0 + \beta_1 x_i, \sigma^2)$, we have
\ba
p(\beta_0, \beta_1, \sigma^2 | \by)
	& \propto p(\by | \beta_0, \beta_1, \sigma^2) 
		\pi(\beta_0, \beta_1, \sigma^2) \\
	& = \prod_{i=1}^n p(y_i | \beta_0, \beta_1, \sigma^2)
		\pi(\beta_0, \beta_1) \pi(\sigma^{-2}) \\
	& \propto \prod_{i=1}^n \frac{1}{\sqrt{2\pi\sigma^2}}
		\exp\left(-\frac{ (y_i - \beta_0 - \beta_1 x_i)^2 }{2\sigma^2} \right)
		\exp\left( - (m_0, m_1) \Sigma^{-1} 
		\begin{pmatrix} m_0 \\ m_1 \end{pmatrix} \right) \\
	& \htab \times (\sigma^{-2})^{a - 1} \exp\left(-b \sigma^{-2}\right)
\ea
First, we have
\ba
p(\beta_0 | \beta_1, \sigma^2, \by) 
	& = \frac{p(\beta_0, \beta_1, \sigma^2 | \by)}{p(\beta_1, \sigma^2 | \by) } \\
	& = \frac{L(\beta_0, \beta_1, \sigma^2 | \by) \pi(\beta_0, \beta_1, \sigma^2)}
		{ L(\beta_1, \sigma^2 | \by) \pi(\beta_1, \sigma^2) }  \\
	& = \frac{p(\beta_0 | \beta_1, \sigma^2, \by) p(\by | \beta_1, \sigma^2) \pi(\beta_0 | \beta_1, \sigma^2)}{p(\beta_1, \sigma^2 | \by)} \\
	& = L(\beta_0 | \beta_1, \sigma^2, \by) \pi(\beta_0 | \beta_1)
\ea
where $\pi(\beta_0 | \beta_1 , \sigma^2) = \pi(\beta_0 | \beta_1) $
by independence of the priors $\pi(\beta_0, \beta_1)$ and $\pi(\sigma^{-2})$. 
But by properties of the normal distribution, we know that
\ba
\beta_0 | \beta_1 \sim \mathcal N \left( m_0 - \frac{v_{01}}{v_{11}} 
	\left( \beta_1 - m_1 \right),
	v_{00} - \frac{v_{01} v_{10}}{v_{11}} \right).
\ea
Similar reasoning yields
\ba
p(\beta_0 | \beta_1, \sigma^2, \by) & \propto 
p(\by | \beta_0, \beta_1, \sigma^2) \pi(\beta_0 | \beta_1) \\
& \sim \mathcal N(\bX \bbeta, \sigma^2)
 \mathcal N \left( m_0 - \frac{v_{01}}{v_{11}} 
	\left( \beta_1 - m_1 \right),
	v_{00} - \frac{v_{01} v_{10}}{v_{11}} \right)
\ea
which can be computed using the result of Problem 2. By symmetry, we also have
\ba
p(\beta_1 | \beta_0, \sigma^2, \by) & \propto
	p(\by | \beta_0, \beta_1, \sigma^2) \pi(\beta_1 | \beta_0) \\
	& \sim \mathcal N(\bX \bbeta, \sigma^2)
		\mathcal N\left( m_1 - \frac{v_{10}}{v_{00}} (\beta_0 - m_0),
		v_{11} - \frac{v_{10} v_{01}}{v_{00}} \right).
\ea
Moreover,
\ba
p(\sigma^{2} | \bbeta, \by) 
	& \propto p(\by | \bbeta, \sigma^2) \pi(\sigma^2 | \bbeta) \\
	& = \frac{ p(\by | \bbeta, \sigma^2) \pi(\sigma^2, \bbeta)}
		{ \pi(\bbeta) } \\
	& = \frac{ p(\by | \bbeta, \sigma^2) \pi(\sigma^{-2}) \pi(\bbeta)}
		{ \pi(\bbeta) } \\
	& = p(\by | \bbeta, \sigma^2) \pi(\sigma^{-2})
\ea
with density proportional to
\ba
\; & (\sigma^{-2})^{n/2} \exp\left(- \frac{1}{2\sigma^2} 
	\left[ (\by - \bX \bbeta)^T (\by - \bX \bbeta) \right] \right)
	\frac{b^a}{\Gamma(a)} (\sigma^{-2})^{a - 1} \exp\left(-b\sigma^{-2}\right) \\
	& = (\sigma^{2})^{-\frac{n}{2} - a + 2 - 1}\exp\left( -\frac{1}{\sigma^2}
		\left[ \frac{1}{2}(\by - \bX \bbeta)^T (\by - \bX \bbeta) + b) \right]
		\right)
\ea
which is the kernel of a InvGamma$\left(a + \frac{n}{2} - 2, b + \norm{\by - \bX \bbeta}_2^2\right)$
for $\sigma^2$.
\item
Here, we see that
\ba
p(\bbeta | \sigma^2, \by) & \propto p(\by | \bbeta, \sigma^2) \pi(\bbeta) \\
	& \propto \exp\left(-\frac{1}{2\sigma^2} 
		(\by - \bX \bbeta)^T (\by - \bX \bbeta) \right)
		\exp\left( -\frac{1}{2} (\bbeta - \bm)^T \bV^{-1} ( \bbeta - \bm )
		 \right) \\
	& \propto 
		\exp\left\{ - \frac{1}{2} \left[
		\bbeta^T \left(\frac{\bX^T \bX}{\sigma^2}\right) \bbeta
		- 2 \bbeta^T \frac{\bX^T \bY}{\sigma^2} \right]
		- \frac{1}{2} \left[ \bbeta^T \bV^{-1} \bbeta
		- 2 \bbeta^T (\bV^{-1} \bm) \right]
		\right\} \\
	& = \exp\left\{ - \frac{1}{2} \left[ \bbeta^T
		\left( \bV^{-1} + \frac{\bX^T \bX}{\sigma^2} \right)^{-1} \bbeta
		- 2 \bbeta^T \left( \bV^{-1} \bm + \frac{\bX^T \bY}{\sigma^2} \right)
		\right]
		\right\} \\
	& = \exp\left\{ - \frac{1}{2} \left[ \bbeta^T
		\bA \bbeta
		+ \bbeta^T \bb
		\right]
		\right\}, 
		\htab \bA = \left(\bV^{-1} + \frac{\bX^T \bX}{\sigma^2} \right)^{-1},
		\bb = -2 \left(\bV^{-1} \bm + \frac{\bX^T \bY}{\sigma^2} \right)\\
	& \propto \exp\left\{ -\frac{1}{2} \left[ 
		\left( \bbeta + \frac{1}{2} \bA^{-1} \bb \right)^T \bA
		\left( \bbeta + \frac{1}{2} \bA^{-1} \bb \right)
		\right]
		\right\} \\
\ea
which is a MVN with 
\ba
\bmu & = - \frac{1}{2} \bA^{-1} \bb \\
	& =  \left(\bV^{-1} + \frac{\bX^T\bX}{\sigma^2} \right)
		\left(\bV^{-1} \bm + \frac{\bX^T \bY}{\sigma^2} \right)
		\bV^{-1} \bV^{-1} \bm + \bV^{-1} \frac{\bX^T\bY}{\sigma^2}
		+ \frac{\bX^T\bY}{\sigma^2} \bV^{-1} \bm 
		+ \frac{1}{\sigma^4} 
\ea
... yikes, this isn't working out.
Given the derived distributions, we simply alternate between sampling from 
$p(\bbeta | \by, \sigma^2)$ and $p(\sigma^{-2} | \bbeta, \by)$, using the current iterates as placeholders. More specifically, we have
\ba
\;&\bbeta^{(0)} \gets \bbeta_0 \\
\;&\sigma^{2, (0)} \gets \sigma_0^2 \\
\;&\text{for } k = 1, \dotsc, K \\
\;&\htab  \bbeta^{(0)} \gets \mathcal N(\bm^\star, \bV^\star | \sigma^{2, (0)},
 \by) \\
\;&\htab  \sigma^{2, (0)} \gets \text{Gamma}
\left(a + \frac{n}{2}, b + \frac{1}{2} (\by - \bX \bbeta^{(0)})^T 
(\by - \bX \bbeta^{(0)})
\bigg| \by
\right) \\
\;&\text{end}\\
\ea


\enumend

\section*{Problem 4}
\enum
\item
The implementation for this problem can be found in Appendix B. Below, we show histograms of the posteriors, after discarding the burn-in of 1,000 samples.
We also show pairwise scatterplots of each parameter. First shown are the plots for the first chain, and the second ones correspond to the second chain. 
\begin{center}
\includegraphics[width=\linewidth]{hists_1.pdf}
\end{center}

\begin{center}
\includegraphics[width=\linewidth]{scat_1.pdf}
\end{center}

\begin{center}
\includegraphics[width=\linewidth]{hists_2.pdf}
\end{center}

\begin{center}
\includegraphics[width=\linewidth]{scat_2.pdf}
\end{center}



\item
% latex table generated in R 3.4.1 by xtable 1.8-2 package
% Fri Nov  3 11:47:18 2017
\begin{table}[ht]
\centering
\begin{tabular}{rrr}
  \hline
10\% & 50\% & 90\% \\ 
  \hline
1.343 & 1.498 & 1.651 \\ 
  0.636 & 0.724 & 0.810 \\ 
  0.520 & 0.622 & 0.757 \\ 
   \hline
\end{tabular}
\end{table}

% latex table generated in R 3.4.1 by xtable 1.8-2 package
% Fri Nov  3 11:47:18 2017
\begin{table}[ht]
\centering
\begin{tabular}{rrr}
  \hline
10\% & 50\% & 90\% \\ 
  \hline
1.341 & 1.501 & 1.658 \\ 
  0.633 & 0.722 & 0.809 \\ 
  0.522 & 0.626 & 0.761 \\ 
   \hline
\end{tabular}
\end{table}

\item
For our first chain, we find that
\ba
\Pr{\beta_1 > 0.5} \approx  0.9998.
\ea
Likewise, for the second chain, we have
\ba
\Pr{\beta_1 > 0.5} \approx 0.9993.
\ea
\item
For our burn-in, we use a general rule of thumb and discard the first 10\% of the samples. Of course, this is dependent on how many samples we have to begin with, so we must investigate trace plots to be thorough. Trace plots for both chains are shown below.
\begin{center}
\includegraphics[width=\linewidth]{trace_1.pdf}
\end{center}

\begin{center}
\includegraphics[width=\linewidth]{trace_2.pdf}
\end{center}


Indeed, both sets of trace plots suggest a well-explored parameter space and a high degree of convergence. 

\enumend

\newpage
\section*{Problem 5}
\enum
\item
Below are the marginal posterior densities running the default \texttt{inla} command.
\begin{center}
\includegraphics[width=\linewidth]{lin_mod_a.pdf}
\end{center}
Moreover, we display a table including the posterior means, standard deviations, and $p$-quantiles, with $p \in \set{0.05, 0.5, 0.975}$. 

 latex table generated in R 3.4.1 by xtable 1.8-2 package
% Thu Nov  2 13:17:01 2017
\begin{table}[ht]
\centering
\begin{tabular}{rrrrr}
  \hline
mean & sd & 0.025quant & 0.5quant & 0.975quant \\ 
  \hline
0.170 & 0.027 & 0.118 & 0.170 & 0.223 \\ 
  -0.036 & 0.005 & -0.047 & -0.036 & -0.026 \\ 
   \hline
\end{tabular}
\end{table}

\item
Note that if $X \sim \text{LogNormal}(\mu, \sigma)$,
then $Y = \log(X) \sim \mathcal N(m, s^2)$. 
\enumend

\newpage
\section*{Appendix: Problem 1 Source}
\lstinputlisting{P1.R}

\section*{Appendix: Problem 4 Source}
\lstinputlisting{P4.R}

\section*{Appendix: Problem 5 Source}
\lstinputlisting{P5.R}

\end{document}


























